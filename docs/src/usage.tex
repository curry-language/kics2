\section{Using the Interactive Environment of \CYS}
\label{sec-kics2}

This section describes the interactive environment
\CYS\index{\CYS}
that supports the development of applications written in Curry.
The implementation of \CYS contains also a separate compiler
which is automatically invoked by the interactive environment.

\subsection{Invoking \CYS}
\label{sec:invoke-kics2}

To start \CYS, execute the command
\ccode{kics2}\pindex{kics2}
(this is a shell script stored in
\code{\cyshome/bin} where \cyshome is the installation directory
of \CYS).
When the system is ready (i.e., when the prompt \ccode{Prelude>} occurs),
the prelude (\code{\cyshome/lib/Prelude.curry})
is already loaded, i.e., all definitions in the prelude are accessible.
Now you can type various commands (see next section)
or an expression to be evaluated.

One can also invoke \CYS with parameters.
These parameters are usual a sequence of commands
(see next section) that are executed before the user interaction
starts. For instance, the invocation
\begin{curry}
kics2 :load Mod :add List
\end{curry}
starts \CYS, loads the main module \code{Mod}, and adds the additional
module \code{List}.
The invocation
\begin{curry}
kics2 :load Mod :eval config
\end{curry}
starts \CYS, loads the main module \code{Mod}, and evaluates
the operation \code{config} before the user interaction starts.
As a final example, the invocation
\begin{curry}
kics2 :load Mod :save :quit
\end{curry}
starts \CYS, loads the main module \code{Mod}, creates an executable,
and terminates \CYS. This invocation could be useful in ``make'' files
for systems implemented in Curry.


\subsection{Commands of \CYS}
\label{sec:kics2-commands}

The {\bf most important commands} of \CYS are
(it is sufficient to type a unique prefix of a command if it is unique,
e.g., one can type \ccode{:r} instead of \ccode{:reload}):

\begin{description}
\item[\fbox{\code{:help}}]\pindex{:help}
Show a list of all available commands.

\item[\fbox{\code{:load $prog$}}]\pindex{:load}
Compile and load the program stored in \code{$prog$.curry}
together with all its imported modules.
% If this file does not exist, the system looks for a FlatCurry
% file \code{$prog$.fcy} and compiles from this intermediate representation.
% If the file \code{$prog$.fcy} does not exists, too, the system looks
% for a file \code{$prog$_flat.xml} containing a FlatCurry program in
% XML representation (compare command \ccode{:xml}\pindex{:xml}),
% translates this into a FlatCurry file \code{$prog$.fcy}
% and compiles from this intermediate representation.

\item[\fbox{\code{:reload}}]\pindex{:reload}
Recompile all currently loaded modules.

\item[\fbox{\code{:add} $m_1 \ldots m_n$}]\pindex{:add}
Add modules $m_1,\ldots,m_n$ to the set of currently loaded modules
so that their exported entities are available in the top-level environment.

\item[\fbox{$expr$}] Evaluate the expression $expr$ to normal form
and show the computed results.
In the default mode, all results of
non-deterministic computations are printed.
One can also print first one result and the next result
only if the user requests it. This behavior can be set by
the option \code{interactive}\index{interactive} (see below).

\textbf{Free variables in initial expressions} must be declared as
in Curry programs.
In order to see the results of their bindings,\footnote{Currently,
bindings are only printed if the initial expression is not an I/O action
(i.e., not of type \ccode{IO\ldots})
and there are not more than ten free variables
in the initial expression.}
they must be introduced by a \ccode{where\ldots{}free} declaration.
For instance, one can write
\begin{curry}
not b where b free
\end{curry}
in order to obtain the following bindings and results:
\begin{curry}
{b = False} True
{b = True} False
\end{curry}
Without these declarations, an error is reported in order to
avoid the unintended introduction of free variables in initial expressions
by typos.

If the free variables in the initial goal are of a polymorphic type,
as in the expression
\begin{curry}
xs++ys=:=[z] where xs,ys,z free
\end{curry}
they are specialized to the type \ccode{()}
(since the current implementation of \CYS does not support
computations with polymorphic logic variables).

\item[\fbox{:eval $expr$}]\pindex{:eval}
Same as $expr$. This command might be useful when putting
commands as arguments when invoking \code{kics2}.

% \item[\fbox{\code{let} $x$ \code{=} $expr$}]
% Define the identifier $x$ as an abbreviation for the expression $expr$
% which can be used in subsequent expressions. The identifier $x$
% is visible until the next \code{load} or \code{reload} command.

\item[\fbox{\code{:quit}}]\pindex{:quit} Exit the system.
\end{description}
%
\bigskip
%
There are also a number of {\bf further commands} that are often
useful:
%
\begin{description}
\item[\fbox{\code{:type $expr$}}]\pindex{:type}
Show the type of the expression $expr$.

\item[\fbox{\code{:programs}}]\pindex{:programs}
Show the list of all Curry programs that are available in the load path.

\item[\fbox{\code{:cd $dir$}}]\pindex{:cd}
Change the current working directory to $dir$.

\item[\fbox{\code{:edit}}]\pindex{:edit}
Load the source code of the current main module into a text editor.
If the variable \code{editcommand} is set in the
configuration file \ccode{.kics2rc}
(see Section~\ref{sec-customization}),
its value is used as an editor command, otherwise
the environment variable \ccode{EDITOR} is used as the editor program.

\item[\fbox{\code{:edit $m$}}]\pindex{:edit}
Load the source text of module $m$ (which must be accessible
via the current load path if no path specification is given)
into a text editor which is defined
as in the command \ccode{:edit}.

\item[\fbox{\code{:show}}]\pindex{:show}
Show the source text of the currently loaded Curry program.
If the variable \code{showcommand} is set in the
configuration file \ccode{.kics2rc}
(see Section~\ref{sec-customization}),
its value is used as a command to show the source text,
otherwise the command \ccode{cat} is used.

\item[\fbox{\code{:show $m$}}]\pindex{:show}
Show the source text of module $m$ which must be accessible
via the current load path if no path specification is given.

\item[\fbox{\code{:source $f$}}]\pindex{:source}
Show the source code of function $f$ (which must be visible
in the currently loaded module) in a separate window.

\item[\fbox{\code{:source $m$.$f$}}]\pindex{:source}
Show the source code of function $f$ defined in module $m$
in a separate window.

\item[\fbox{\code{:browse}}]\pindex{:browse}
Start the CurryBrowser to analyze the currently loaded
module together with all its imported modules
(see Section~\ref{sec-currybrowser} for more details).

\item[\fbox{\code{:interface}}]\pindex{:interface}
Show the interface of the currently loaded
module, i.e., show the names of all imported modules,
the fixity declarations of all exported operators,
the exported datatypes declarations and the types
of all exported functions.

\item[\fbox{\code{:interface $m$}}]\pindex{:interface}
Similar to \ccode{:interface}
but shows the interface of the module $m$
which must be in the load path of \CYS.

\item[\fbox{\code{:usedimports}}]\pindex{:usedimports}
Show all calls to imported functions in the currently loaded module.
This might be useful to see which import declarations are really necessary.

\item[\fbox{\code{:set $option$}}]\pindex{:set}
Set or turn on/off a specific option
of the \CYS environment (see \ref{sec:options} for a description
of all options). Options are turned on by the prefix
\ccode{+} and off by the prefix \ccode{-}. Options that can only
be set (e.g., \code{path}) must not contain a prefix.

\item[\fbox{\code{:set}}]\pindex{:set}
Show a help text on the possible options
together with the current values of all options.

\item[\fbox{\code{:save}}]\pindex{:save} Save the currently loaded
program as an executable evaluating the main expression \ccode{main}.
The executable is stored in the file \code{Mod}
if \code{Mod} is the name of the currently loaded main module.

\item[\fbox{\code{:save $expr$}}]\pindex{:save} Similar as \ccode{:save}
but the expression $expr$ (typically: a call to the main
function) will be evaluated by the executable.

\item[\fbox{\code{:fork $expr$}}]\pindex{:fork}
The expression $expr$, which is typically of type \ccode{IO ()},
is evaluated in an independent process which runs in
parallel to the current \CYS process.
All output and error messages from this new process are suppressed.
This command is useful to test distributed Curry programs
where one can start
a new server process by this command. The new process
will be terminated when the evaluation of the expression $expr$
is finished.

\item[\fbox{\code{:!$cmd$}}]\pindex{:"!} Shell escape: execute $cmd$ in a Unix shell.

\end{description}

\subsection{Options of \CYS}
\label{sec:options}

The following options (which can be set by the command \ccode{:set})
are currently supported:

\begin{description}
\item[\fbox{\code{path $path$}}]\pindex{path} Set the additional search path
for loading modules to $path$.
Note that this search path is only used for loading modules
inside this invocation of \CYS.

The path is a list of directories separated by \ccode{:}.
The prefix \ccode{\char126} is replaced by the home directory as
in the following example:
\begin{curry}
:set path aux:~/tests
\end{curry}
Relative directory names are replaced by absolute ones
so that the path is independent of later changes of the
current working directory.

\item[\fbox{\code{bfs}}]\pindex{bfs}
Set the search mode to evaluate non-deterministic expressions
to breadth-first search.
This is the default search strategy.
Usually, all non-deterministic values
are enumerated and printed with a breadth-first strategy, but one can also
print only the first value or all values by interactively requesting them
(see below for these options).

\item[\fbox{\code{dfs}}]\pindex{dfs}
Similarly to \code{bfs} but use a depth-first search strategy
to compute and print the values of the initial expression.

\item[\fbox{\code{ids}}]\pindex{ids}
Similarly to \code{bfs} but use an iterative-deepening strategy
to compute and print the values of the initial expression.
The initial depth bound is 100 and the depth-bound is
doubled after each iteration.

\item[\fbox{\code{ids $n$}}]\pindex{ids}
Similarly to \code{ids} but use an initial depth bound of $n$.

\item[\fbox{\code{parallel}}]\pindex{parallel}
Similarly to \code{bfs} but use a parallel search strategy
to compute and print the values of the initial expression.
The system chooses an appropriate number
of threads according the current number of available processors.

\item[\fbox{\code{parallel $n$}}]\pindex{parallel}
Similarly to \code{parallel} but use $n$ parallel threads.

\item[\fbox{\code{prdfs }}]\pindex{prdfs}
Set the search mode to evaluate non-deterministic expressions
to primitive depth-first search. This is usually the fastest method
to print \emph{all} non-deterministic values.
However, it does not support the evaluation of values
by interactively requesting them.

\item[\fbox{\code{choices $n$}}]\pindex{choices}
Show the internal choice structure
(according to the implementation described in
\cite{BrasselHanusPeemoellerReck11})
resulting from the complete evaluation of the main expression
in a tree-like structure.
This mode is only useful for debugging or
understanding the implementation of non-deterministic
evaluations used in \CYS.
If the optional argument $n$ is provided,
the tree is shown up to depth $n$.

\item[\fbox{\code{supply $i$}}]\pindex{supply}
(not available in global installations, see Section~\ref{sec-install})
Use implementation $i$ as the identifier supply for choice structures
(see \cite{BrasselHanusPeemoellerReck11} for a detailed explanation).
Currently, the following values for $i$ are supported:
\begin{description}
\item[\code{integer}:]\pindex{integer}
Use unbounded integers as choice identifiers.
This implementation is described in \cite{BrasselHanusPeemoellerReck11}.
\item[\code{ghc}:]\pindex{ghc}
Use a more sophisticated implementation of choice
identifiers (based on the ideas described in \cite{AugustssonRittriSynek94})
provided by the Glasgow Haskell Compiler.
\item[\code{pureio}:]\pindex{pureio}
Use IO references (i.e., memory cells) for choice identifiers.
This is the most efficient implementation for top-level depth-first search
but cannot be used for more sophisticated search methods
like encapsulated search.
\item[\code{ioref} (default):]\pindex{ioref}
Use a mixture of \code{ghc} and \code{pureio}.
IO references are used for top-level depth-first search
and \code{ghc} identifiers are used for encapsulated search methods.
\end{description}

\item[\fbox{\code{v$n$}}]\pindex{v}\index{verbosity}
Set the verbosity level to $n$. The following values are allowed
for $n$:
\begin{description}
\item[$n=0$:] Do not show any messages (except for errors).
\item[$n=1$:] Show only messages of the front-end, like loading
of modules.
\item[$n=2$:]
Show also messages of the back end, like compilation messages
from the Haskell compiler.
\item[$n=3$:]
Show also intermediate messages and commands of the compilation
process.
\item[$n=4$:]
Show also all intermediate results of the compilation process.
\end{description}

\item[\fbox{\code{prompt $p$}}]\pindex{prompt}
Sets the user prompt which is shown when \CYS is waiting for input.
If the parameter $p$ starts with a letter or a percent sign,
the prompt is printed as the given parameter,
where the sequence \ccode{\%s} is expanded to the list of
currently loaded modules
and \ccode{\%\%} is expanded to a percent sign.
If the prompt starts with a double quote, it is read as a string and,
therefore, also supports the normal escape sequences that can occur
in Curry programs. The default setting is
\begin{curry}
:set prompt "%s> "
\end{curry}

\item[\fbox{\code{+/-interactive}}]\pindex{interactive}
Turn on/off the interactive mode.
In the interactive mode, the next non-deterministic value
is only computed when the user requests it.
Thus, one has also the possibility to terminate the
enumeration of all values after having seen some values.

\item[\fbox{\code{+/-first}}]\pindex{first}
Turn on/off the first-only mode.
In the first-only mode, only the first value
of the main expression is printed (instead of all values).

\item[\fbox{\code{+/-optimize}}]\pindex{optimize}
Turn on/off the optimization of the target program.

\item[\fbox{\code{+/-bindings}}]\pindex{bindings}
Turn on/off the binding mode.
If the binding mode is on (default),
then the bindings of the free variables of the initial expression
are printed together with the result of the expression.

\item[\fbox{\code{+/-time}}]\pindex{time}
Turn on/off the time mode. If the time mode is on,
the cpu time and the elapsed time
of the computation is always printed together with the result
of an evaluation.

\item[\fbox{\code{+/-trace}}]\pindex{trace}
Turn on/off the trace mode. If the trace mode is on,
it is possible to trace the sources of failing computations.

\item[\fbox{\code{+/-profile}}]\pindex{profile}
(only available when configured during installation, see Section~\ref{sec-install})
Turn on/off the profile mode. If the profile mode is on,
expressions as well as programs are compiled with GHC's profiling
capabilities enabled. For expressions, evaluation will automatically
generate a file \code{Main.prof} containing the profiling information
of the evaluation.
For compiled programs, the profiling has to be manually activated
using runtime options when executed:
\begin{curry}
kics2 :set +profile :load MyProgram.curry :save :quit
./MyProgram +RTS -p -RTS [additional arguments]
\end{curry}

\item[\fbox{\code{+/-ghci}}]\pindex{ghci}
Turn on/off the ghci mode.
In the ghci mode, the initial goal is send to the interactive version
of the Glasgow Haskell Compiler. This might result in a slower
execution but in a faster startup time since the linker
to create the main executable is not used.

\item[\fbox{\code{safe}}]\pindex{safe}
Turn on the safe execution mode.
In the safe execution mode, the initial goal is
not allowed to be of type \code{IO} and the program should not
import the module \code{Unsafe}.
Furthermore, the allowed commands are
\code{eval}, \code{load}, \code{quit}, and \code{reload}.
This mode is useful to use \CYS in uncontrolled environments,
like a computation service in a web page, where \CYS could
be invoked by
\begin{curry}
kics2 :set safe
\end{curry}

\item[\fbox{\code{parser $opts$}}]\pindex{parser}
Define additional options passed to the \CYS front end, i.e.,
the parser program \code{\cyshome/bin/cymake}.
For instance, setting the option
\begin{curry}
:set parser -F --pgmF=transcurry
\end{curry}
has the effect that each Curry module to be compiled is
transformed by the preprocessor command \code{transcurry}
into a new Curry program which is actually compiled.

\item[\fbox{\code{cmp $opts$}}]\pindex{cmp}
Define additional options passed to the \CYS compiler.
For instance, setting the option
\begin{curry}
:set cmp -O 0
\end{curry}
has the effect that all optimizations performed by the \CYS compiler
are turned off.

\item[\fbox{\code{ghc $opts$}}]\pindex{ghc}
Define additional options passed to the Glasgow Haskell Compiler (GHC)
when the generated Haskell programs are compiled.
Many options necessary to compile Curry programs
are already set (you can see them by setting the verbosity
level to 2 or greater).
One has to be careful when providing additional options.
For instance, in a global installation of \CYS
(see Section~\ref{sec-install}), libraries are pre-compiled
so that inconsistencies might occur if compilation options
might be changed.

It is safe to pass specific GHC linking options.
For instance, to enforce the static linking of libraries
in order to generate an executable (see command \ccode{:save})
that can be executed in another environment, one could set the
options
\begin{curry}
:set ghc -static -optl-static -optl-pthread
\end{curry}

Other options are useful for experimental purposes,
but those should be used only in local installations
(see Section~\ref{sec-install}) to avoid inconsistent
target codes for different libraries.
For instance, setting the option
\begin{curry}
:set ghc -DDISABLE_CS
\end{curry}
has the effect that the constraint store used to enable
an efficient access to complex bindings is disabled.
Similarly,
\begin{curry}
:set ghc -DSTRICT_VAL_BIND
\end{curry}
has the effect that expressions in a unification constraint
(\code{=:=}) are always fully evaluated
(instead of the evaluation to a head normal form only)
before unifying both sides.
Since these options influence the compilation of the run-time system,
one should also enforce the recompilation of Haskell programs
by the GHC option \ccode{-fforce-recomp}, e.g., one should set
\begin{curry}
:set ghc -DDISABLE_CS -fforce-recomp
\end{curry}

\item[\fbox{\code{rts $opts$}}]\pindex{rts}
Define additional run-time options passed to the executable
generated by the Glasgow Haskell Compiler, i.e., the parameters
\ccode{+RTS $o$ -RTS} are passed to the executable.
For instance, setting the option
\begin{curry}
:set rts -H512m
\end{curry}
has the effect that the minimum heap size is set to 512 megabytes.

\item[\fbox{\code{args $arguments$}}]\pindex{args}
Define run-time arguments passed to the executable
generated by the Glasgow Haskell Compiler.
For instance, setting the option
\begin{curry}
:set args first second
\end{curry}
has the effect that the I/O operation \code{getArgs}
(see library \code{System} (Section~\ref{Library:System})
returns the value \code{["first","second"]}.

\end{description}


\subsection{Source-File Options}

If the evaluation of operations in some main module loaded into
\CYS requires specific options, like an iterative-deepening
search strategy, one can also put these options into the source
code of this module in order to avoid setting these options
every time when this module is loaded.
Such
{\bf source-file options}\index{source-file option}\index{option!in source file}
must occur before the module header, i.e., before the first declaration
(module header, imports, fixity declaration, defining rules, etc)
occurring in the module.
Each source file option must be in a line of the form
\begin{curry}
{-# KiCS2_OPTION $opt$ #-}
\end{curry}
where $opt$ is an option that can occur in a \ccode{:set} command
(compare Section~\ref{sec:options}).
Such a line in the source code (which is a comment according to
the syntax of Curry)
has the effect that this option is set by the \CYS command
\ccode{:set $opt$} whenever this module is loaded (not reloaded!)
as a main module. For instance, if a module starts with the
lines
\begin{curry}
{-# KiCS2_OPTION ids #-}
{-# KiCS2_OPTION +ghci #-}
{-# KiCS2_OPTION v2 #-}
module M where
$\ldots$
\end{curry}
then the load command \ccode{:load M} will also
set the options for iterative deepening, using \code{ghci}
and verbosity level 2.


\subsection{Using \CYS in Batch Mode}

Although \CYS is primarily designed as an interactive system,
it can also be used to process data in batch mode.
For example, consider a Curry program, say \code{myprocessor}, that
reads argument strings from the command line and processes them.
Suppose the entry point is a function called \code{just_doit}
that takes no arguments. Such a processor can be invoked from
the shell as follows:
\begin{curry}
> kics2 :set args string1 string2 :load myprocessor.curry :eval just_doit :quit
\end{curry}
The \ccode{:quit} directive in necessary to avoid \CYS going
into interactive mode after the
excution of the expression being evaluated.
The actual run-time arguments (\code{string1}, \code{string2})
are defined by setting the option \code{args} (see above).

Here is an example to use \CYS in this way:
\begin{curry}
> kics2 :set args Hello World :add System :eval "getArgs >>= putStrLn . unwords" :quit
Hello World
>
\end{curry}


\subsection{Command Line Editing}

In order to have support for line editing or history functionality
in the command line of \CYS (as often supported by the \code{readline}
library), you should have the Unix command \code{rlwrap} installed
on your local machine.
If \code{rlwrap} is installed, it is used by \CYS if called on a terminal.
If it should not be used (e.g., because it is executed
in an editor with \code{readline} functionality), one can
call \CYS with the parameter \ccode{--noreadline}
(which must occur as the first parameter).


\subsection{Customization}
\label{sec-customization}

In order to customize the behavior of \CYS to your own preferences,
there is a configuration file which is read by \CYS when it is invoked.
When you start \CYS for the first time, a standard version of
this configuration file is copied with the name
\ccode{.kics2rc}\pindex{kics2rc}\pindex{.kics2rc}
into your home directory. The file contains definitions
of various settings, e.g., about showing warnings, using Curry extensions,
programs etc.
After you have started \CYS for the first time, look into this file
and adapt it to your own preferences.


\subsection{Emacs Interface}

Emacs is a powerful programmable editor suitable for program development.
It is freely available for many platforms
(see \url{http://www.emacs.org}).
The distribution of \CYS contains also a special
\emph{Curry mode}\index{Curry mode}\index{Emacs}
that supports the development of Curry programs in
the Emacs environment.
This mode includes support for syntax highlighting,
finding declarations in the current buffer, and
loading Curry programs into \CYS
in an Emacs shell.

The Curry mode has been adapted from a similar mode for Haskell programs.
Its installation is described in the file \code{README}
in directory \ccode{\cyshome/tools/emacs} which also contains
the sources of the Curry mode and a short description about
the use of this mode.

%%% Local Variables:
%%% mode: latex
%%% TeX-master: "manual"
%%% End:
