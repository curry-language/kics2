\section{Auxiliary Files}
\label{sec-auxfiles}

During the translation and execution of a Curry program with \CYS,
various intermediate representations of the source program are created
and stored in different files which are shortly explained in this section.
If you use \CYS, it is not necessary to know about
these auxiliary files because they are automatically generated
and updated. You should only remember the command for deleting
all auxiliary files (\ccode{cleancurry}, see Section~\ref{sec-general})
to clean up your directories.

Usually, the auxiliary files are invisible:
if the Curry module $M$ is stored in directory $dir$,
the corresponding auxiliary files are stored in directory
\ccode{$dir$/.curry/kics2-$v$} where $v$ is the version of \CYS.
Thus, the auxiliary files produced by different versions of \CYS
causes no conflicts.
This scheme is also used for hierarchical module names:
if the module $D1.D2.M$ is stored in directory $dir$
(i.e., the module is actually stored in $dir/D1/D2/M.curry$),
then the corresponding Prolog program is stored in directory
\ccode{$dir$/.curry/kics2-$v$/D1/D2}.

\medskip

The various components of \CYS create the following auxiliary files.
\begin{description}
\item[\code{prog.fcy}:] This file contains the Curry program
in the so-called ``FlatCurry'' representation where all functions are global
(i.e., lambda lifting has been performed) and pattern matching
is translated into explicit case/or expressions
(compare Appendix~\ref{sec-flatcurry}).
This representation might be useful for other back ends and
compilers for Curry and is the basis doing meta-programming in Curry.
This file is implicitly
generated when a program is compiled with \CYS.
It can be also explicitly generated by the
front end of \CYS:\pindex{kics2 frontend}
\begin{curry}
kics2 frontend --flat -i$\cyshome$/lib prog
\end{curry}
The FlatCurry representation of a Curry program is usually
generated by the front-end after parsing, type checking and eliminating
local declarations.

\item[\code{prog.fint}:] This file contains the interface
of the program in the so-called ``FlatCurry'' representation,
i.e., it is similar to \code{prog.fcy} but contains only exported
entities and the bodies of all functions omitted (i.e., ``external'').
This representation is useful for providing a fast access
to module interfaces.
This file is implicitly generated when a program is compiled with \CYS
and stored in the same directory as \code{prog.fcy}.

\item[\code{Curry_prog.hs}:] This file contains a Haskell program
as the result of translating the Curry program with the
\CYS compiler.

If the Curry module $M$ is stored in the directory $dir$,
the corresponding Haskell program is stored in the directory
\ccode{$dir$/.curry/kics2}.
This is also the case for hierarchical module names:
if the module $D1.D2.M$ is stored in the directory $dir$
(i.e., the module is actually stored in $dir/D1/D2/M.curry$),
then the corresponding Haskell program is stored in
\ccode{$dir$/.curry/kics2/D1/D2/Curry_prog.hs}.

\item[\code{Curry_prog.hi}:] This file contains the interface
of the Haskell program \code{Curry_prog.hs} when the latter program is
compiled in order to execute it.
This file is stored in the same directory as \code{Curry_prog.hs}.

\item[\code{Curry_prog.o}:] This file contains the object code
of the Haskell program \code{Curry_prog.hs} when the latter program is
compiled in order to execute it.
This file is stored in the same directory as \code{Curry_prog.hs}.

\item[\code{Curry_prog.nda}:] This file contains some information about
the determinism behavior of operations that is used by the
\CYS compiler (see \cite{BrasselHanusPeemoellerReck11}
for more details about the use of this information).
This file is stored in the same directory as \code{Curry_prog.hs}.

\item[\code{Curry_prog.info}:] This file contains some
information about the top-level functions of module \code{prog}
that are used by the interactive environment,
like determinism behavior or IO status.
This file is stored in the same directory as \code{Curry_prog.hs}.

\item[\code{prog}:] This file contains the executable
after compiling and saving a program with \CYS
(see command \ccode{:save} in Section~\ref{sec:kics2-commands}).

\end{description}

%%% Local Variables:
%%% mode: latex
%%% TeX-master: "manual"
%%% End:
