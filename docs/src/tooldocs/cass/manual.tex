\section{CASS: A Generic Curry Analysis Server System}
\label{sec-cass}

CASS\index{CASS}\index{analyzing programs}\index{program!analysis}
(Curry Analysis Server System)
is a tool for the analysis of Curry programs.
CASS is generic so that various kinds of analyses (e.g., groundness,
non-determinism, demanded arguments) can be easily integrated into CASS.
In order to analyze larger applications consisting of dozens or hundreds
of modules, CASS supports a modular and incremental analysis of
programs. Moreover, it can be used by different programming tools,
like documentation generators, analysis environments, program
optimizers, as well as Eclipse-based development environments. For
this purpose, CASS can also be invoked as a server system to get a
language-independent access to its functionality. CASS is completely
implemented Curry as a master/worker architecture to exploit parallel
or distributed execution environments.
The general design and architecture of CASS is described
in \cite{HanusSkrlac14}.
In the following, CASS is presented from a perspective
of a programmer who is interested to analyze Curry programs.

\subsection{Installation}

The current implementation of CASS is a package
managed by the Curry Package Manager CPM.
Thus, to install the newest version of CASS, use the following commands:
%
\begin{curry}
> cypm update
> cypm install cass
\end{curry}
%
This downloads the newest package, compiles it, and places
the executable \code{cass} into the directory \code{\$HOME/.cpm/bin}.
Hence it is recommended to add this directory to your path
in order to execute CASS as described below.

\subsection{Using CASS to Analyze Programs}

CASS is intended to analyze various operational properties
of Curry programs. Currently, it contains more than a dozen
program analyses for various properties.
Since most of these analyses are based on abstract interpretations,
they usually approximate program properties.
To see the list of all available analyses, use the help option of CASS:
\begin{curry}
> cass -h
Usage: $\ldots$
$\vdots$
Registered analyses names:
$\ldots$
Demand          : Demanded arguments
Deterministic   : Deterministic operations
$\vdots$
\end{curry}
More information about the meaning of the various analyses
can be obtained by adding the short name of the analysis:
\begin{curry}
> cass -h Deterministic
$\ldots$
\end{curry}
For instance, consider the following Curry module \code{Rev.curry}:
\begin{curry}
append :: [a] -> [a] -> [a]
append []     ys = ys
append (x:xs) ys = x : append xs ys

rev :: [a] -> [a]
rev []     = []
rev (x:xs) = append (rev xs) [x]

fromTo :: Int -> Int -> [Int]
fromTo m n | m <= n    = m : fromTo (m+1) n
           | otherwise = []

main :: Int -> Int -> [Int]
main m n = rev (fromTo m n)
\end{curry}
%
CASS supports three different usage modes to analyze this program.

\subsubsection{Batch Mode}

In the batch mode, CASS is started as a separate application
via the shell command \code{cass},
where the analysis name and the name of the module to be analyzed
must be provided:\footnote{More output is generated when
the parameter \code{debugLevel} is changed in the configuration file
\code{.curryanalysisrc} which is installed in the user's home directory
when CASS is started for the first time.}
\begin{curry}
> cass Demand Rev
append : demanded arguments: 1
fromTo : demanded arguments: 1,2
main : demanded arguments: 1,2
rev : demanded arguments: 1
\end{curry}
The \code{Demand} analysis shows the list of argument positions
(e.g., 1 for the first argument) which are demanded in order
to reduce an application of the operation to some constructor-rooted value.
Here we can see that both arguments of \code{main} are demanded
whereas only the first argument of \code{append} is demanded.
This information could be used in a Curry compiler
to produce more efficient target code.

The batch mode is useful to test a new analysis and get the information
in human-readable form so that one can experiment
with different abstractions or analysis methods.

\subsubsection{API Mode}

The API mode is intended to use analysis information in some
application implemented in Curry. Since CASS is implemented in Curry,
one can import the modules of the CASS implementation and
use the CASS interface operations to start an analysis and use the
computed results. For instance, CASS provides an operation
(defined in the module \code{CASS.Server})
\begin{curry}
analyzeGeneric :: Analysis a -> String -> IO (Either (ProgInfo a) String)
\end{curry}
to apply an analysis (first argument) to some module (whose name is
given in the second argument). The result is either the analysis
information computed for this module or an error message in case of
some execution error.

In order to use CASS via the API mode in a Curry program,
one has to use the package \code{cass} by the Curry package manager CPM
(the subsequent explanation assumes familiarity with the basic
features of CPM):
\begin{enumerate}
\item
Add the dependency on package \code{cass} and also on
package \code{cass-analysis}, which contains some base definitions,
in the package specification file \code{package.json}.
\item
Install these dependencies by \ccode{cypm install}.
\end{enumerate}
Then you can import in your application the modules
provided by CASS.

The module \code{Analysis.ProgInfo} (from package \code{cass-analysis})
contains operations to access the analysis information computed by CASS.
For instance, the operation
\begin{curry}
lookupProgInfo:: QName -> ProgInfo a -> Maybe a
\end{curry}
returns the information about a given qualified name in the
analysis information, if it exists.
As a simple example, consider the demand analysis which is implemented
in the module \code{Analysis.Demandedness} by the following operation:
\begin{curry}
demandAnalysis :: Analysis DemandedArgs
\end{curry}
\code{DemendedArgs} is just a type synonym for \code{[Int]}.
We can use this analysis in the following simple program:
\begin{currynomath}
import CASS.Server           ( analyzeGeneric )
import Analysis.ProgInfo     ( lookupProgInfo )
import Analysis.Demandedness ( demandAnalysis )

demandedArgumentsOf :: String -> String -> IO [Int]
demandedArgumentsOf modname fname = do
  deminfo <- analyzeGeneric demandAnalysis modname >>= return . either id error
  return $ maybe [] id (lookupProgInfo (modname,fname) deminfo)
\end{currynomath} %$
Of course, in a realistic program, the program analysis
is performed only once and the computed information \code{deminfo}
is passed around to access it several times.
Nevertheless, we can use this simple program to compute the
demanded arguments of \code{Rev.main}:
\begin{curry}
$\ldots$> demandedArgumentsOf "Rev" "main"
[1,2]
\end{curry}


\subsubsection{Server Mode}

The server mode of CASS can be used in an application implemented in
some language that does not have a direct interface to Curry.
In this case, one can connect to CASS via
some socket using a simple communication protocol that is specified
in the file \code{Protocol.txt} (in package \code{cass})
and sketched below.

To start CASS in the server mode, one has to execute the command
\begin{curry}
> cass --server [ -p <port> ]
\end{curry}
where an optional port number for the communication can be
provided. Otherwise, a free port number is chosen and shown. In the
server mode, CASS understands the following commands:
\begin{curry}
GetAnalysis
SetCurryPath <dir1>:<dir2>:...
AnalyzeModule          <analysis name> <output type> <module name>
AnalyzeInterface       <analysis name> <output type> <module name>
AnalyzeFunction        <analysis name> <output type> <module name> <function name>
AnalyzeDataConstructor <analysis name> <output type> <module name> <constructor name>
AnalyzeTypeConstructor <analysis name> <output type> <module name> <type name>
StopServer
\end{curry}
The output type can be \code{Text}, \code{CurryTerm}, or \code{XML}.
The answer to each request can have two formats:
\begin{curry}
error <error message>
\end{curry}
if an execution error occured, or
\begin{curry}
ok <n>
<result text>
\end{curry}
where \code{<n>} is the number of lines of the result text.
For instance, the answer to the command \code{GetAnalysis}
is a list of all available analyses. The list has the form
\begin{curry}
<analysis name> <output type>
\end{curry}
For instance, a communication could be:
\begin{curry}
> GetAnalysis
< ok 5
< Deterministic CurryTerm
< Deterministic Text
< Deterministic XML
< HigherOrder   CurryTerm
< DependsOn     CurryTerm
\end{curry}
The command \code{SetCurryPath} instructs CASS to use the given
directories to search for modules to be analyzed. This is necessary
since the CASS server might be started in a different location than
its client.

Complete modules are analyzed by \code{AnalyzeModule}, whereas
\code{AnalyzeInterface} returns only the analysis information of exported
entities. Furthermore, the analysis results of individual functions,
data or type constructors are returned with the remaining analysis
commands. Finally, \code{StopServer} terminates the CASS server.

For instance, if we start CASS by
\begin{curry}
> cass --server -p 12345
\end{curry}
we can communicate with CASS as follows (user inputs are prefixed by \ccode{>});
\begin{curry}
> telnet localhost 12345
Connected to localhost.
> GetAnalysis
ok 57
Overlapping XML
Overlapping CurryTerm
Overlapping Text
Deterministic XML
...
> AnalyzeModule Demand Text Rev
ok 4
append : demanded arguments: 1
fromTo : demanded arguments: 1,2
main : demanded arguments: 1,2
rev : demanded arguments: 1
> AnalyzeModule Demand CurryTerm Rev
ok 1
[(("Rev","append"),"demanded arguments: 1"),(("Rev","fromTo"),"demanded arguments: 1,2"),(("Rev","main"),"demanded arguments: 1,2"),(("Rev","rev"),"demanded arguments: 1")]
> AnalyzeModule Demand XML Rev
ok 24
<?xml version="1.0" standalone="yes"?>

<results>
  <operation>
    <module>Rev</module>
    <name>append</name>
    <result>demanded arguments: 1</result>
  </operation>
  <operation>
    <module>Rev</module>
    <name>fromTo</name>
    <result>demanded arguments: 1,2</result>
  </operation>
  <operation>
    <module>Rev</module>
    <name>main</name>
    <result>demanded arguments: 1,2</result>
  </operation>
  <operation>
    <module>Rev</module>
    <name>rev</name>
    <result>demanded arguments: 1</result>
  </operation>
</results>
> StopServer
ok 0
Connection closed by foreign host.
\end{curry}


\subsection{Implementing Program Analyses}

Each program analysis accessible by CASS must be registered
in the CASS module \code{CASS.Registry}.
An analysis to be registered must contain an operation of type
\begin{curry}
Analysis a
\end{curry}
where \code{a} denotes the type of analysis results.
For instance, the \code{Overlapping} analysis is implemented as a function
\begin{curry}
overlapAnalysis :: Analysis Bool
\end{curry}
where the Boolean analysis result indicates whether a Curry operation
is defined by overlapping rules.

In order to add a new analysis to CASS, one has to implement
a corresponding analysis operation, registering it in the module
\code{CASS.Registry} (in the constant \code{registeredAnalysis})
and compile/install the modified CASS implementation.

An analysis is implemented as a mapping from Curry programs
represented in FlatCurry into the analysis result.
Hence, to implement the \code{Overlapping} analysis, we define
the following operation on function declarations in FlatCurry format:
\begin{curry}
import FlatCurry.Types
$\ldots$
isOverlappingFunction :: FuncDecl -> Bool
isOverlappingFunction (Func _ _ _ _ (Rule _ e))   = orInExpr e
isOverlappingFunction (Func f _ _ _ (External _)) = f == ("Prelude","?")

-- Check an expression for occurrences of Or:
orInExpr :: Expr -> Bool
orInExpr (Var _)       = False
orInExpr (Lit _)       = False
orInExpr (Comb _ f es) = f == (pre "?") || any orInExpr es
orInExpr (Free _ e)    = orInExpr e
orInExpr (Let bs e)    = any orInExpr (map snd bs) || orInExpr e
orInExpr (Or _ _)      = True
orInExpr (Case _ e bs) = orInExpr e || any orInBranch bs
 where orInBranch (Branch _ be) = orInExpr be
orInExpr (Typed e _)   = orInExpr e
\end{curry}
%
In order to enable the inclusion of different analyses in CASS,
CASS offers several constructor operations for the abstract type \ccode{Analysis a}
(defined in the CASS module \code{Analysis}).
Each analysis has a name provided as a first argument
to these constructors. The name is used to store the
analysis information persistently and to pass specific analysis tasks
to analysis workers.
For instance, a simple function analysis which depends only on a
given function definition can be defined by the
analysis constructor
\begin{curry}
simpleFuncAnalysis :: String -> (FuncDecl -> a) -> Analysis a
\end{curry}
The arguments are the analysis name and the actual analysis function.
Hence, the ``overlapping rules'' analysis can be specified as
\begin{curry}
import Analysis.Types
$\ldots$
overlapAnalysis :: Analysis Bool
overlapAnalysis = simpleFuncAnalysis "Overlapping" isOverlappingFunction
\end{curry}
Another analysis constructor supports the definition
of a function analysis with dependencies (which is implemented via a fixpoint
computation):
\begin{curry}
dependencyFuncAnalysis :: String -> a -> (FuncDecl -> [(QName,a)] -> a)
                       -> Analysis a
\end{curry}
Here, the second argument specifies the start value of the fixpoint analysis,
i.e., the bottom element of the abstract domain.

For instance, a determinism analysis could be based
on an abstract domain described by the data type
\begin{curry}
data Deterministic = NDet | Det
\end{curry}
Here, \code{Det} is interpreted as ``the operation always evaluates
in a deterministic manner on ground constructor terms.''
However, \code{NDet} is interpreted as ``the operation \emph{might}
evaluate in different ways for given ground constructor terms.''
The apparent imprecision is due to the approximation of the analysis.
For instance, if the function \code{f} is defined by overlapping rules
and the function \code{g} \emph{might} call \code{f}, then \code{g}
is judged as non-deterministic (since it is generally undecidable
whether \code{f} is actually called by \code{g} in some run of the
program).

The determinism analysis requires to examine the current
function as well as all directly or indirectly called functions
for overlapping rules.
Due to recursive function definitions, this analysis cannot be done
in one shot---it requires a fixpoint computation.
CASS provides such fixpoint computations and requires
only the implementation of an operation of type
\begin{curry}
FuncDecl -> [(QName,a)] -> a
\end{curry}
where \ccode{a} denotes the type of abstract values.
The second argument of type \code{[(QName,a)]}
represents the currently known analysis values
for the functions \emph{directly} used in this function declaration.

In our example, the determinism analysis can be implemented
by the following operation:
\begin{curry}
detFunc :: FuncDecl -> [(QName,Deterministic)] -> Deterministic
detFunc (Func f _ _ _ (Rule _ e)) calledFuncs =
  if orInExpr e || freeVarInExpr e || any (==NDet) (map snd calledFuncs)
    then NDet
    else Det
\end{curry}
Thus, it computes the abstract value \code{NDet}
if the function itself is defined by overlapping rules or
contains free variables that might cause non-deterministic guessing
(we omit the definition of \code{freeVarInExpr} since it is quite
similar to \code{orInExpr}), or
if it depends on some non-deterministic function.

The complete determinism analysis can be specified as
\begin{curry}
detAnalysis :: Analysis Deterministic
detAnalysis = dependencyFuncAnalysis "Deterministic" Det detFunc
\end{curry}
This definition is sufficient to execute the analysis
with CASS since the analysis system takes care of computing fixpoints,
calling the analysis functions with appropriate values,
analyzing imported modules, etc.
Nevertheless, the analysis must be defined so that the fixpoint
computation always terminates. This can be achieved by using
an abstract domain with finitely many values and ensuring
that the analysis function is monotone w.r.t.\ some ordering on the values.
