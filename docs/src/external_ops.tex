\section{External Operations}
\label{sec:external-operations}

\index{operation!external}\index{external operation}
Currently, \CYS has no general interface to external operations,
i.e., operations whose semantics is not defined by program rules
in a Curry program but by some code written in another programming
language.
Thus, if an external operation should be added
to the system, this operation must be declared as \code{external}
in the Curry source code
and an implementation for this external operation
must be provided for the run-time system.
An external operation is defined as follows in the Curry source code:
\begin{enumerate}
\item
Add a type declaration for the external operation somewhere
in a module defining this operation (usually, the prelude
or some system module).
\item
For external operations it is not allowed to define any
rule since their semantics is determined by an external implementation.
Instead of the defining rules, you have to write
\begin{curry}
f external
\end{curry}
below the type declaration for the external operation \code{f}.
\end{enumerate}
Furthermore, an implementation of the external operation
must be provided in the target language of the \CYS compiler,
i.e., in Haskell, and inserted in the compiled code.
In order to simplify this task, \CYS follows some code
conventions that are described in the following.

Assume you want to implement your own concatenation
for strings in a module \code{String}.
The name and type of this string concatenation should be
\begin{curry}
sconc :: String -> String -> String
\end{curry}
Since the primitive Haskell implementation of this operation
does not now anything about the operational mechanism of Curry
(e.g., needed narrowing, non-deterministic rewriting),
the arguments need to be completely evaluated before
the primitive implementation is called.
This can be easily obtained by the prelude operation \code{(\$\#\#)}
that applies an operation to the \emph{normal form} of the given
argument, i.e., this operation evaluates the argument
to its normal form before applying the operation to it.\footnote{%
There is also a similar prelude operation \code{(\$\#)}
which evaluates the argument only to head-normal form.
This is a bit more efficient and can be used for unstructured
types like \code{Bool}.}
Thus, we define \code{sconc} by
\begin{currynomath}
sconc :: String -> String -> String
sconc s1 s2 = (prim_sconc $## s1) $## s2

prim_sconc :: String -> String -> String
prim_sconc external
\end{currynomath}
so that it is ensured that the external operation \code{prim_sconc}
is always called with complete evaluated arguments.

In order to define the Haskell code implementing \code{prim_sconc},
one has to satisfy the naming conventions of \CYS.
The \CYS compiler generates the following code for the
external operation \code{prim_sconc} (note that the generated
Haskell code for the module \code{String} is stored in the file
\code{.curry/kics2/Curry_String.hs}):
\begin{curry}
d_C_prim_sconc :: Curry_Prelude.OP_List Curry_Prelude.C_Char
               -> Curry_Prelude.OP_List Curry_Prelude.C_Char
               -> ConstStore
               -> Curry_Prelude.OP_List Curry_Prelude.C_Char
d_C_prim_sconc x1 x2 x3500 = external_d_C_prim_sconc x1 x2 x3500
\end{curry}
The type constructors \code{OP_List} and \code{C_Char}
of the prelude \code{Curry_Prelude}\footnote{Note that all
translated Curry modules are imported in the Haskell code fully qualified
in order to avoid name conflicts.}
correspond to the Curry type constructors for lists and characters.
The Haskell operation \code{external_d_C_prim_sconc}
is the external operation to be implemented in Haskell by the programmer.
The additional argument of type \code{ConstStore}
represents the current set of constraints when this
operation is called. This argument is intended to provide
a more efficient access to binding constraints and can be
ignored in standard operations.

If \code{String.curry} contains the code
of the Curry function \code{sconc} described above,
the Haskell code implementing the external operations
occurring in the module \code{String} must be in the
file \code{External_String.hs} which is located in the same
directory as the file \code{String.curry}.
The \CYS compiler appends the code contained in
\code{External_String.hs} to the generated code
stored in the file \code{.curry/kics2/Curry_String.hs}.\footnote{%
If the file \code{External_String.hs} contains also
some import declarations at the beginning, these import declarations
are put after the generated import declarations.}

In order to complete our example, we have to write into the
file \code{External_String.hs} a definition of the Haskell function
\code{external_d_C_prim_sconc}.
Thus, we start with the following definitions:
\begin{curry}
import qualified Curry_Prelude as CP

external_d_C_prim_sconc :: CP.OP_List CP.C_Char -> CP.OP_List CP.C_Char
                        -> ConstStore -> CP.OP_List CP.C_Char
\end{curry}
First, we import the standard prelude with the name \code{CP}
in order to shorten the writing of type declarations.
In order to write the final code of this operation,
we have to convert the Curry-related types
(like \code{C_Char}) into the corresponding Haskell types (like \code{Char}).
Note that the Curry-related types contain information about
non-deterministic or constrained values
(see \cite{BrasselHanusPeemoellerReck11,BrasselHanusPeemoellerReck11WLP})
that are meaningless in Haskell.
To solve this conversion problem, the implementation of \CYS
provides a family of operations to perform these conversions
for the predefined types occurring in the standard prelude.
For instance, \code{fromCurry} converts a Curry type into the
corresponding Haskell type, and \code{toCurry} converts
the Haskell type into the corresponding Curry type.
Thus, we complete our example with the definition
(note that we simply ignore the final argument representing the
constraint store)
\begin{curry}
external_d_C_prim_sconc s1 s2 _ =
  toCurry ((fromCurry s1 ++ fromCurry s2) :: String)
\end{curry}
Here, we use Haskell's concatenation operation \ccode{++}
to concatenate the string arguments.
The type annotation \ccode{:: String} is necessary
because \ccode{++} is a polymorphic function
so that the type inference system of Haskell
has problems to determine the right instance of the conversion
function.

The conversion between Curry types and Haskell types,
i.e., the family of conversion operation \code{fromCurry}
and \code{toCurry}, is defined in the \CYS implementation
for all standard data types.
In particular, it is also defined on function types so that one can
easily implement external Curry I/O actions by using
Haskell I/O actions.
For instance, if we want to implement
an external operation to print some string as an output line,
we start by declaring the external operations in the Curry module \code{String}:
\begin{currynomath}
printString :: String -> IO ()
printString s = prim_printString $## s

prim_printString :: String -> IO ()
prim_printString external
\end{currynomath} % $
Next we add the corresponding implementation in the file
\code{External_String.hs} (where \code{C_IO} and \code{OP_Unit} are the names
of the Haskell representation of the Curry type constructor \code{IO}
and the Curry data type \ccode{()}, respectively):
\begin{curry}
external_d_C_prim_printString :: CP.OP_List CP.C_Char -> ConstStore
                              -> CP.C_IO CP.OP_Unit
external_d_C_prim_printString s _ = toCurry putStrLn s
\end{curry}
Here, Haskell's I/O action \code{putStrLn} of type \ccode{String -> IO ()}
is transformed into a Curry I/O action \ccode{toCurry putStrLn}
which has the type
\begin{curry}
CP.OP_List CP.C_Char -> CP.C_IO CP.OP_Unit
\end{curry}
When we compile the Curry module \code{String},
\CYS combines these definitions in the target program so that
we can immediately use the externally defined operation
\code{printString} in Curry programs.

As we have seen, \CYS transforms a name like
\code{primOP} of an external operation into the name
\code{external_d_C_primOP} for the Haskell operation
to be implemented, i.e., only a specific prefix is added.
However, this is only valid if no special characters occur
in the Curry names.
Otherwise (in order to generate a correct Haskell program),
special characters are translated into specific
names prefixed by \ccode{OP_}. For instance,
if we declare the external operation
\begin{curry}
(<&>) :: Int -> Int -> Int
(<&>) external
\end{curry}
the generated Haskell module contains the code
\begin{curry}
d_OP_lt_ampersand_gt :: Curry_Prelude.C_Int -> Curry_Prelude.C_Int
                     -> ConstStore -> Curry_Prelude.C_Int
d_OP_lt_ampersand_gt x1 x2 x3500 = external_d_OP_lt_ampersand_gt x1 x2 x3500
\end{curry}
so that one has to implement the operation
\code{external_d_OP_lt_ampersand_gt} in Haskell.
If in doubt, one should look into the generated Haskell code
about the names and types of the operations to be implemented.

Finally, note that this method to connect functions implemented in Haskell
to Curry programs provides the opportunity to connect
also operations written in other programming languages
to Curry via Haskell's foreign function interface.