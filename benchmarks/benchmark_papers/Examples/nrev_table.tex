%%%%%%%%%%%%%%%%%%%%%%%%%%%%%%%%%%%%%%%%%%%%%%%%%%%%%%%%%%%%%%%%%%%%%
% Example for executable benchmark paper:
%
% The paper contains all executable code to benchmark the normalization
% time of the Curry system for the naive reverse benchmark.
% The results are formatted as a LaTeX table.
%%%%%%%%%%%%%%%%%%%%%%%%%%%%%%%%%%%%%%%%%%%%%%%%%%%%%%%%%%%%%%%%%%%%%

\documentclass{article}

\usepackage{currycode}
\usepackage{graphicx}

\begin{document}
\sloppy

\section{Introduction}

This is a short LaTeX document showing some features of an
Executable Benchmark Paper.

%%% The benchmark program:
\begin{curryprog}
----------------------------------------------------------------------
import Benchmarks
import BenchmarkGoodies

-- Naive reverse implementation:
nrev []     = []
nrev (x:xs) = (nrev xs) ++ [x]

-- Naive reverse benchmark: we benchmark the time of the complete naive
-- reverse and the list construction and create a difference
-- benchmark to obtain the time for the naive reverse only:
nrevBenchDiff :: Int -> Benchmark Float
nrevBenchDiff n =
  benchTimeNF (return (nrev [1..n]))  .-.  benchTimeNF (return [1..n])

nrevMainBench =
  execBench (mapBench (map (\ (n,f) -> (show n, showF2 f)))
               (runOn (\n -> 1 *> nrevBenchDiff n) [500,1000..2500])) >>=
    return . benchInputsResultsAsTable ("length:","time:")
----------------------------------------------------------------------
\end{curryprog}

\section{Benchmarks}

\paragraph{Note:}
%
All benchmarks were executed on a \runcurry{getOS} machine
(named ``\runcurry{getHostName}'')
running \runcurry{getSystemID} \runcurry{getSystemRelease}
(with an ``\runcurry{getCPUModel}'' processor containing
\runcurry{getCoreNumber} cores).

\subsection{Nrev Benchmark}

\begin{center}
\runcurry{nrevMainBench}
\end{center}


\end{document}
