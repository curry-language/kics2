\vspace*{5cm}
\begin{center}
\textbf{Zusammenfassung}
\end{center}
\par
Ein Constraint-Programming-System besteht �blicherweise aus zwei Komponenten: einer Modellierungssprache zur Spezifikation eines Constraint-Problems und einer L�sungskomponente, die durch Anwendung spezieller Algorithmen ein gegebenes Constraint-Modell l�st. Die funktional-logische Programmiersprache Curry ist aufgrund ihres deklarativen Stils gut f�r die Einbettung einer Constraint-Modellierungssprache geeignet. Allerdings ist die direkte Realisierung eines zustandsbehafteten Constraint-Solvers in Curry aufgrund der Seiteneffektfreiheit dieser Sprache schwierig. Diese Arbeit beschreibt, wie man die KiCS2-Curry-Implementierung, die Curry-Programme in rein funktionale Haskell-Programme �bersetzt, durch Integration eines funktionalen Constraint-Programming-Frameworks um eine Bibliothek zur Modellierung und L�sung von Finite-Domain-Constraints erweitern kann. Schlie�lich wird von den konkreten Finite-Domain-Constraints und -Solvern abstrahiert und gezeigt, wie man generische Schnittstellen zur Einbettung von Constraints und FD-Solvern in KiCS2 implementiert.

\cleardoublepage


