\section{Evaluation}
In diesem Kapitel soll die vorgestellte Erweiterung von KiCS2 um eine Finite-Domain-Constraint-Bibliothek und die Integration extern implementierter FD-Solver mit Hilfe einiger Benchmarks evaluiert werden.
\\
Alle Benchmarks wurden auf einem PC mit Intel Core 2 Duo (3 GHz) Prozessor und 3 GB Arbeitsspeicher unter Ubuntu 12.04 ("'Precise Pangolin"') durchgef�hrt. Dabei wurde KiCS2 Version 0.2 mit dem Glasgow Haskell Compiler (GHC Version 7.4.1) ausgef�hrt. Mit Hilfe der \lstinline|time|-Option von KiCS2 wurde jeweils die Ausf�hrungszeit in Sekunden gemessen und dann der Mittelwert von drei Messdurchl�ufen berechnet. Falls ein Benchmark-Programm innerhalb von 10 Minuten kein Ergebnis geliefert hat, wird dies bei den Messergebnissen durch "'n.a."' kenntlich gemacht.
\\
Zun�chst wurde getestet, inwieweit sich das "'nach oben"' Verschieben von \lstinline|Guard|-Ausdr�cken mit Bindungs-Constraints im KiCS2-Auswertungsbaum sowie das zus�tzliche Ablaufen dieses Baumes zur Suche nach \lstinline|WrappedConstraint|s mittels \lstinline|searchWrappedCs| auf die Auswertung von Ausdr�cken auswirkt, die keine Finite-Domain-Constraints enthalten. Dazu wurden die Benchmarks f�r die Unifikation, die KiCS2 zur Verf�gung stellt (\emph{UnificationBench.curry}), einmal mit und einmal ohne die CLPFD-Erweiterung ausgef�hrt. Zu den Benchmark-Programmen geh�ren \lstinline|last| (Berechnung des letzten Elements einer Liste), \lstinline|grep| (Matching eines Wortes mit einem regul�ren Ausdruck), \lstinline|halfPeano| (Halbierung einer Peano-Zahl), \lstinline|varInExp| (Suche nach einer Variable in einem arithmetischen Ausdruck), \lstinline|simplify| (Vereinfachung eines arithmetischen Ausdrucks), \lstinline|palindrome| (�berpr�ft, ob eine Liste ein Palindrom ist) und \lstinline|horseMan| (L�sung einer Gleichung bez�glich der K�pfe und Beine von Pferden und Menschen):
\begin{figure}[!h]
\begin{center}
\begin{tabular}{|l|*{2}{r|}}
\hline
Benchmark  & KiCS2 & KiCS2 + CLPFD  \\\hline
\lstinline|last|       &   1.18  &   1.18  \\
\lstinline|grep|       &   1.14  &   1.24  \\
\lstinline|halfPeano|  &  53.17  &  54.1   \\
\lstinline|varInExp|   &   2.01  &   n.a.  \\
\lstinline|simplify|   &  54.56  &  95.75  \\
\lstinline|palindrome| &  64.3   &  85.93  \\
\lstinline|horseMan|   &  11.58  &  14.79  \\
\hline
\end{tabular}
\caption{Benchmarks: Unifikation in KiCS2 ohne und mit CLPFD-Erweiterung}
\end{center}
\end{figure}
\\
Wie man an den Ergebnissen sieht, ben�tigt die erweiterte KiCS2-Version in fast allen F�llen l�nger als die urspr�ngliche Version. Dies h�ngt vermutlich in erster Linie mit dem "'Hochziehen"' der \lstinline|Guard|-Ausdr�cke mit Bindungs-Constraints im KiCS2-Auswertungsbaum zusammen. Besonders extrem wirkt sich dies auf den \lstinline|varInExp|-Benchmark aus, bei dem versucht wird, die einzige Variable in einem tief verschachtelten arithmetischen Ausdruck (25.000 Knoten) mit Hilfe der Unifikation zu finden.
\\
Auch die Vereinfachung eines arithmetischen Ausdrucks mit 4003 Knoten ben�tigt fast doppelt so lange, wenn man die erweiterte KiCS2-Version verwendet. Bei den �brigen Benchmarks bleibt der Overhead f�r die CLPFD-Erweiterung hingegen im Rahmen.
\par
In einem weiteren Benchmark wurde die Performance der Finite-Domain-Solver anhand des N-Damen-Problems verglichen. Betrachtet wurden dabei eine generate\&test-Variante des N-Damen-Problems ohne FD-Constraints (siehe Anhang \ref{anhangE}), die L�sung des CLPFD-Modells (vergleiche Listing \ref{nDamen}) mit den beiden MCP-FD-Solvern, wobei jeweils einmal die \lstinline|InOrder|- und einmal die \lstinline|FirstFail|-Labeling-Strategie getestet wurden, und der direkte Aufruf des Gecode-Solvers ohne KiCS2 mit dem im Grundlagenkapitel vorgestellten MCP-Modell (vergleiche Listing \ref{mcpNDamen}).
\begin{figure}[!h]
\begin{center}
\begin{tabular}{|*{7}{r|}}
\hline
n  & gen. \& test & \multicolumn{2}{c|}{Overton} & \multicolumn{2}{c|}{Gecode} & Gecode (direkt) \\\hline
   &           & \lstinline|InOrder|   & \lstinline|FirstFail| & \lstinline|InOrder|   & \lstinline|FirstFail| & \lstinline|FirstFail| \\\hline
%1  &  $<$0.01  &  $<$0.01  &  $<$0.01  &  $<$0.01  &  $<$0.01  &  $<$0.01  \\
%2  &  $<$0.01  &  $<$0.01  &  $<$0.01  &  $<$0.01  &  $<$0.01  &  $<$0.01  \\
%3  &  $<$0.01  &  $<$0.01  &  $<$0.01  &  $<$0.01  &  $<$0.01  &  $<$0.01  \\
4  &     0.04  &  $<$0.01  &  $<$0.01  &  $<$0.01  &  $<$0.01  &  $<$0.01  \\
5  &     0.66  &     0.02  &     0.02  &     0.02  &     0.02  &     0.01  \\
6  &    11.12  &     0.03  &     0.03  &     0.02  &     0.02  &     0.02  \\
7  &   225.99  &     0.13  &     0.10  &     0.07  &     0.04  &     0.03  \\
8  &     n.a.  &     0.48  &     0.38  &     0.18  &     0.09  &     0.07  \\
9  &     n.a.  &     2.28  &     1.82  &     0.74  &     0.29  &     0.20  \\
10 &     n.a.  &    10.16  &     8.45  &     2.64  &     0.86  &     0.65  \\
11 &     n.a.  &    51.62  &    43.14  &    13.16  &     3.58  &     2.80  \\
12 &     n.a.  &   301.06  &   248.87  &    83.95  &    18.91  &    14.26  \\
\hline
\end{tabular}
\caption{Benchmarks: Performance der Solver im Vergleich}
\end{center}
\end{figure}
\\
Die Benchmarks zeigen, dass die generate\&test-Implementierung des N-Damen-Problems f�r gr�\-�ere n nicht mit den FD-Solvern konkurrieren kann. Bereits die Berechnung des 8-Damen-Problems ben�tigt l�nger als zehn Minuten. Die MCP-FD-Solver l�sen das gleiche Problem in unter einer Sekunde. Auch f�r gr��ere n-Werte berechnen beide Solver die L�sungen in einer akzeptablen Zeit.
\\
Wie zu erwarten war, ist der auf C++-basierende Gecode-Solver allerdings erheblich schneller als der in Haskell realisierte Overton-Solver. Durch Einsatz einer vorteilhaften Labeling-Strategie wie \lstinline|FirstFail|, bei der die Labeling-Variablen mit dem am weitesten eingeschr�nkten Wertebereich bevorzugt werden, k�nnen die Zeiten f�r beide MCP-Solver noch einmal erheblich verbessert werden.
\\
Vergleicht man den direkten Aufruf des Gecode-Solvers �ber das MCP-Framework mit dem Aufruf dieses Solvers �ber KiCS2, so zeigt sich, dass die Performance fast identisch ist. Erst bei der L�sung gr��erer Modelle wirkt sich der zus�tzliche �bersetzungsschritt, der in KiCS2 durchgef�hrt werden muss, auf die Ausf�hrungszeiten aus.

