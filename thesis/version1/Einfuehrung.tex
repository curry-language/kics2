\section{Einf�hrung}
\label{chapter_einfuehrung}
Constraint-Programming befasst sich mit der L�sung von Problemen durch Spezifikation von Eigenschaften und Bedingungen (= Constraints), die von m�glichen L�sungen dieses Problems erf�llt werden m�ssen. Man definiert diese Bedingungen deklarativ durch das Aufstellen von Regeln f�r Variablen, die einen endlichen oder unendlichen Wertebereich besitzen. Die Modellierung eines Problems mit endlichem Wertebereich wird auch als Finite-Domain-Constraint-Programming bezeichnet.
\par
Ein Constraint-Programming-System besteht im wesentlichen aus zwei Elementen: einer Modellierungskomponente, mir deren Hilfe das zu l�sende Problem beschrieben wird, und einer L�sungskomponente, einem sogenannten Constraint-Solver, der nach L�sungen f�r das beschriebene Problem sucht. Dazu speichert, kombiniert und vereinfacht der Solver die Constraints eines Modells mit Hilfe spezieller Algorithmen. Mit Hilfe der Labeling-Technik "'probiert"' er Variablenbelegungen aus und propagiert diese (Constraint-Propagierung), um auf diese Weise den Wertebereich weiterer Constraint-Variablen einzuschr�nken. Diese Techniken werden iterativ angewandt, bis entweder eine g�ltige Belegungen f�r alle Constraint-Variablen gefunden wurde oder bis ein Widerspruch auftritt. In diesem Fall "'springt"' der Solver in einen konsistenten Zustand zur�ck und probiert eine andere Belegungen aus (Backtracking).
\par
Zu den Anwendungsgebieten des Constraint-Programmings z�hlen die Erstellung von Stunden-, Fahr- und Personaleinsatzpl�nen. Neben logischen Sprachen wie PROLOG, die spezielle Constraint-Programming-Bibliotheken zur Verf�gung stellen, gibt es auch eine ganze Reihe von Softwaresystemen zur Modellierung und L�sung von Constraint-Problemen. Beispiele hierf�r sind die auf C++-basierende Solver-Bibliothek Gecode (Generic Constraint Development Environment \cite{Gecode}) oder das in der Objekt-orientierten Programmiersprache Java realisierte TAILOR-Tool (\cite{Gent:2007:TSC:1770681.1770699}), das in der Solver-unabh�ngigen Modellierungssprache Essence' modellierte Constraint-Probleme mit Hilfe des Minion oder Gecode-Solvers l�st.
\\
Im Bereich der funktionalen Programmierung haben Tom Schrijvers, Peter Stuckey und Philip Wadler mit ihrem Monadic-Constraint Programming-Framework (kurz: MCP-Framework, \cite{234095}) gezeigt, wie man einen Finite-Domain-Constraint-Solver mit Hilfe von Monaden in der seiteneffektfreien Sprache Haskell implementieren kann. 
\par
Auf den ersten Blick ist auch die funktional-logische Programmiersprache Curry gut f�r die Einbettung einer Constraint-Modellierungssprache geeignet: So unterst�tzt Curry die Programmierung mit freien Variablen und bietet einen deklarativen Programmierstil. Allerdings gilt in Curry das Prinzip der referentiellen Transparenz, es handelt sich also um eine zustandslose, seiteneffektfreie Sprache. Dies erschwert die direkte Realisierung stark zustandsbehafteter Constraint-Solver in Curry. 
\\
Diese Arbeit hat sich zum Ziel gesetzt, die KiCS2-Curry-Implementierung, die funktional-logische Programme in rein funktionale Haskell-Programme �bersetzt, um die M�glichkeit zur Programmierung mit Finite-Domain-Constraints zu erweitern. Dazu soll eine Finite-Domain-Constraint-Bibliothek f�r KiCS2 entwickelt werden. Gel�st werden sollen diese Constraints mit Hilfe der FD-Solver des Monadic-Constraint-Programming-Frameworks (\cite{234095}), die daher in die KiCS2-Implemen\-tierung integriert werden sollen. Schlie�lich sollen die wichtigsten Erkenntnisse aus dieser Integration abgeleitet werden und in die Entwicklung generischer Schnittstellen zur Einbindung weiterer Constraints und Constraint-Solver in KiCS2 flie�en. In einem letzten Schritt sollen diese Schnittstellen beispielhaft f�r die zuvor entwickelte FD-Constraint-Bibliothek und die MCP-FD-Solver implementiert werden.
\par
Das n�chste Kapitel liefert eine kurze �bersicht �ber die wichtigsten Grundlagen, welche zum Verst�ndnis dieser Arbeit erforderlich sind. Dazu werden in einzelnen Unterkapiteln die Programmiersprache Curry, die KiCS2-Curry-Implementierung sowie das Monadic-Constraint-Pro\-gramming-Framework vorgestellt. Kapitel 3 beschreibt die grundlegende Idee f�r die Implementierung von Finite-Domain-Constraints in KiCS2. In Kapitel 4 wird die Implementierung vorgestellt. Es ist in drei gr��ere Unterabschnitte unterteilt, die sich mit der Entwicklung einer CLPFD-Bibliothek f�r KiCS2, der Integration der Finite-Domain-Solver des MCP-Frameworks sowie der Entwicklung generischer Schnittstellen zur Integration weiterer Constraints und Constraint-Solver befassen. Kapitel 5 evaluiert die hier vorgestellte Implementierung mit Hilfe geeigneter Benchmarks. Und das letzte Kapitel liefert eine Zusammenfassung sowie einen Ausblick auf m�gliche Weiterentwicklungen.




\clearpage
