\section{MCP-Solver-Implementierung}
\label{anhangC}
Wie schon im Kapitel 4.3.2 angedeutet, verwendet man zur �bersetzung der Constraints in die MCP-FD-Modellierungssprache unterschiedliche Implementierungen f�r die beiden Constraint-Solver. Genauer gesagt, die Implementierung variiert f�r die beiden Solver hinsichtlich der �berset\-zung von FD-Listen in MCP-Collections. Der Gecode-Solver erwartet, dass man f�r jede FD-Liste eine MCP-Collection-Variable einf�hrt und etwaigen konstanten Werten in dieser Liste mit Hilfe der \lstinline|(@!!)|-Funktion des \lstinline|(@=)|-Constraints eine Position in der durch die Variable repr�sentierten Collection zuweist (vergleiche hierzu die Funktion \lstinline|newColCs|). Der Overton-Solver kann hingegen auch mit konstanten MCP-Collections, also mit Listen vom Typ \lstinline|ModelInt| arbeiten.
\\
Um neue MCP-Collection-Variablen f�r den Gecode-Solver einf�hren zu k�nnen, ist es notwendig, FD-Listen eindeutig identifizieren zu k�nnen. Nur auf diese Weise kann gew�hrleistet werden, dass bei der �bersetzung die gleiche FD-Liste auf die gleiche MCP-Collection-Variable abgebildet wird. Daher wird der unten angegebene Datentyp zur Repr�sentation von FD-Listen eingef�hrt. F�r diese FD-Listen werden dann bei der �bersetzung nach dem gleichen Prinzip, das auch schon zur �bersetzung von FD-Variablen angewandt wurde, neue MCP-Collection-Variablen eingef�hrt (siehe dazu die Funktionen \lstinline|translateGecodeList| und \lstinline|newColVar|).

\begin{lstlisting}[language=Haskell,basicstyle=\footnotesize\ttfamily,caption=Einf�hrung eines Identifiers f�r FD-Listen]
-- representation of lists of fd terms:
-- ID to identify a specific fd list is necessary for translating
-- fd lists into mcp collections for the Gecode solver
data FDList a = FDList ID [a]

-- helper function to convert curry integer lists to lists of fd terms
toFDList :: Constrainable a b => ID -> CP.OP_List a -> FDList b
toFDList i vs = FDList i (toFDList' vs)
  where
   toFDList' CP.OP_List        = []
   toFDList' (CP.OP_Cons v vs) = toCsExpr v : toFDList' vs
\end{lstlisting}

\begin{lstlisting}[language=Haskell,basicstyle=\footnotesize\ttfamily,caption=Angepasste \lstinline|FDConstraint|s]
data FDConstraint
  = FDRel RelOp (FDTerm Int) (FDTerm Int)
  | FDArith ArithOp (FDTerm Int) (FDTerm Int) (FDTerm Int)
  | FDSum (FDList (FDTerm Int)) (FDTerm Int)
  | FDAllDifferent (FDList (FDTerm Int))
  | FDDomain (FDList (FDTerm Int)) (FDTerm Int) (FDTerm Int)
  | FDLabeling LabelingStrategy (FDList (FDTerm Int)) ID
 deriving (Eq,Show,Typeable)
\end{lstlisting}

\newpage
\textbf{MCPSolver-Modul:}

\begin{lstlisting}[language=Haskell,basicstyle=\footnotesize\ttfamily,escapechar=�,frame=no]
{-# LANGUAGE TypeFamilies, FlexibleContexts #-}

module MCPSolver where

import Types
import FDData
import ExternalSolver
import qualified Curry_Prelude as CP

import Data.Expr.Sugar
import Control.CP.ComposableTransformers (solve)
import Control.CP.SearchTree (addC, Tree (..),MonadTree(..))
import Control.CP.EnumTerm
import Control.CP.FD.Model
import Control.CP.FD.FD (FDInstance, FDSolver, getColItems)
import Control.CP.FD.Interface (colList)
import Control.CP.FD.Solvers
import Control.CP.FD.Gecode.Common (GecodeWrappedSolver)
import Control.CP.FD.Gecode.Runtime (RuntimeGecodeSolver)
import Control.CP.FD.OvertonFD.OvertonFD
import Control.CP.FD.OvertonFD.Sugar

import qualified Data.Map as Map
import Control.Monad.State
import Data.Maybe (fromJust)
import Data.List ((\\))

-- ---------------------------------------------------------------------------
-- WrappableConstraint instance for FDConstraint
-- ---------------------------------------------------------------------------
instance WrappableConstraint FDConstraint where
  updateVars = updateFDConstr updateFDVar

-- ---------------------------------------------------------------------------
-- ExternalFDSolver instance for MCP Solvers
-- ---------------------------------------------------------------------------
instance ExternalFDSolver MCPSolver FDConstraint where
  newtype SolverModel MCPSolver FDConstraint = ModelWrapper [Model]

  -- |Type for storing labeling information for the MCP solvers:
  -- @labelVars    - labeling variables in original representation
  -- @domainVars   - list of fd variables, for which a domain was defined
  --                 necessary to check, whether a domain was defined for the 
  --                 labeling variables
  -- @mcpLabelVars - labeling variables translated into corresponding 
  --                 MCP representation
  -- @labelID      - fresh ID, necessary for constructing choices over solutions,
  --                 when transforming solver solutions into binding constraints
  -- @strategy     - labeling strategy
  data LabelInfo MCPSolver FDConstraint = 
    Info { labelVars    :: Maybe (FDList (FDTerm Int))
         , domainVars   :: [FDTerm Int]
         , mcpLabelVars :: Maybe ModelCol
         , labelID      :: Maybe ID
         , strategy     :: Maybe LabelingStrategy }

  newtype Solutions MCPSolver FDConstraint = 
    SolWrapper (SolutionInfo CP.C_Int (FDTerm Int))

  translate Overton fdCs = translateOverton fdCs
  translate Gecode  fdCs = translateGecode fdCs

  solveWith = solveWithMCP 

  makeCsSolutions _ (SolWrapper solutions) e = bindSolutions solutions e

-- type synonyms for easier access to associated types
type MCPModel     = SolverModel MCPSolver FDConstraint
type MCPLabelInfo = LabelInfo MCPSolver FDConstraint
type MCPSolution  = Solutions MCPSolver FDConstraint

-- ---------------------------------------------------------------------------
-- Translation to MCP model
-- ---------------------------------------------------------------------------
-- Stores MCP representation of constraint variables
-- @key   - Integer value provided by (getKey i) (i :: ID)
-- @value - MCP representation of constraint variable with ID i 
type IntVarMap = Map.Map Integer ModelInt

-- Stores MCP representation of lists of constraint variables
-- @key   - Integer value provided by (getKey i) (i :: ID)
-- @value - MCP representation of list of constraint variables with ID i 
type ColVarMap = Map.Map Integer ModelCol

-- Translation state for Haskell's state monad
-- @intVarMap     - Table of already translated constraint variables
-- @colVarMap     - Table of already translated lists of constraint variables 
--                  (only used for translateGecode)
-- @nextIntVarRef - Next variable reference
-- @nextColVarRef - Next list variable reference (only used for translateGecode)
-- @additionalCs  - additional constraints for MCP collections 
--                  (only used for translateGecode)
-- @labelInfo     - collected labeling information
data TLState = TLState { 
  intVarMap     :: IntVarMap,
  colVarMap     :: ColVarMap,
  nextColVarRef :: Int,
  nextIntVarRef :: Int,
  additionalCs  :: [Model],
  labelInfo     :: MCPLabelInfo }

-- Initial state
baseTLState :: TLState
baseTLState = TLState { 
  intVarMap     = Map.empty,
  colVarMap     = Map.empty,
  nextIntVarRef = 0,
  nextColVarRef = 0,
  additionalCs  = [],
  labelInfo     = baseLabelInfo }

-- Initial (empty) labeling information
baseLabelInfo :: MCPLabelInfo
baseLabelInfo = Info {
  labelVars    = Nothing,
  domainVars   = [],
  mcpLabelVars = Nothing,
  labelID      = Nothing,
  strategy     = Nothing }

-- The Overton and Gecode solvers work on different representations of lists of 
-- constraint variables. Therefore each solver has its own translation function:

-- Translates list of finite domain constraints into a MCP model for the
-- Overton Solver and collects labeling information if available
-- using Haskell's state monad
translateOverton :: [FDConstraint] -> (MCPModel,MCPLabelInfo)
translateOverton fdCs = 
  let (mcpCs,state) = runState (mapM (translateConstr translateOvertonList) fdCs) 
                        baseTLState
      info          = labelInfo state
  in (ModelWrapper mcpCs, info)

-- Translates list of finite domain constraints into a MCP model for the
-- Gecode Solver and collects labeling information if available
-- using Haskell's state monad
translateGecode :: [FDConstraint] -> (MCPModel,MCPLabelInfo)
translateGecode fdCs = 
  let (mcpCs,state) = runState (mapM (translateConstr translateGecodeList) fdCs) 
                        baseTLState
      info          = labelInfo state
      mcpColCs      = additionalCs state
  in (ModelWrapper (mcpCs ++ mcpColCs), info)

-- Translates a single finite domain constraint into a specific MCP constraint
-- using Haskell's state monad.
-- This function works for both solvers by calling different functions to 
-- translate lists
-- @tlList - function to translate lists of constraint variables 
--           to MCP collections
translateConstr :: (FDList (FDTerm Int) -> State TLState ModelCol) -> FDConstraint 
                -> State TLState Model
translateConstr _ (FDRel relop t1 t2)                  = do 
  mcpTerm1 <- translateTerm t1
  mcpTerm2 <- translateTerm t2
  let mcpRelop = translateRelOp relop
  return $ mcpRelop mcpTerm1 mcpTerm2
translateConstr _ (FDArith arithOp t1 t2 r)            = do 
  mcpTerm1  <- translateTerm t1
  mcpTerm2  <- translateTerm t2
  mcpResult <- translateTerm r
  let mcpArithOp = translateArithOp arithOp
  return $ (mcpArithOp mcpTerm1 mcpTerm2) @= mcpResult
translateConstr tlList (FDSum vs r)                    = do 
  mcpVs     <- tlList vs
  mcpResult <- translateTerm r
  return $ (xsum mcpVs) @= mcpResult
translateConstr tlList (FDAllDifferent vs)             = do 
  mcpVs <- tlList vs
  return $ allDiff mcpVs
translateConstr tlList (FDDomain vs@(FDList _ ts) l u) = do 
  mcpVs <- tlList vs
  mcpL  <- translateTerm l
  mcpU  <- translateTerm u
  state <- get
  let info     = labelInfo state
      dVars    = domainVars info
      newInfo  = info { domainVars = dVars ++ ts }
      newState = state { labelInfo = newInfo }
  put newState
  return $ domain mcpVs mcpL mcpU
translateConstr tlList (FDLabeling str vs j)           = do
  mcpVs <- tlList vs
  state <- get
  let info     = labelInfo state
      newInfo  = info { labelVars    = Just vs
                      , mcpLabelVars = Just mcpVs
                      , labelID      = Just j
                      , strategy     = Just str }
      newState = state { labelInfo = newInfo }
  put newState 
  return (toBoolExpr True)

-- Constraining a MCP collection of variables to a domain
-- defined by a lower and upper boundary
domain :: ModelCol -> ModelInt -> ModelInt -> Model
domain varList lower upper = forall varList (\var -> var @: (lower,upper))

-- Translates integer terms to appropriate MCP terms
-- using Haskell's state monad
translateTerm :: FDTerm Int -> State TLState ModelInt
translateTerm (Const x) = return (cte x)
translateTerm v@(FDVar i) = do 
  state <- get
  let varMap = intVarMap state
  maybe (newVar v) return (Map.lookup (getKey i) varMap)

-- Creates a new MCP variable for the given constraint variable
-- Updates the translation state by inserting the MCP representation
-- of the variable into the map and incrementing the varref counter
newVar :: FDTerm Int -> State TLState ModelInt
newVar (FDVar i) = do 
  state <- get
  let varMap   = intVarMap state
      varRef   = nextIntVarRef state
      nvar     = asExpr (ModelIntVar varRef :: ModelIntTerm ModelFunctions)
      newState = state { nextIntVarRef = varRef + 1
                       , intVarMap = Map.insert (getKey i) nvar varMap }
  put newState
  return nvar

-- Translates lists of fd terms to MCP collection for the Overton Solver
translateOvertonList :: FDList (FDTerm Int) -> State TLState ModelCol
translateOvertonList (FDList _ vs) = do mcpExprList <- mapM translateTerm vs
                                        return (list mcpExprList)

-- Translates lists of fd terms to MCP collection for the Gecode Solver
translateGecodeList :: FDList (FDTerm Int) -> State TLState ModelCol
translateGecodeList l@(FDList i vs) = do 
  state <- get
  let varMap = colVarMap state
  maybe (newColVar l) return (Map.lookup (getKey i) varMap)

-- Creates a new MCP collection variable for the given list,
-- Updates the translation state by inserting its MCP representation
-- into the map and incrementing the corresponding varref counter,
-- Creates additional constraints for the collection variable
-- describing its size and elements (only used for translateGecode)
newColVar :: FDList (FDTerm Int) -> State TLState ModelCol
newColVar (FDList i vs) = do 
  mcpVs <- mapM translateTerm vs
  state <- get
  let varMap   = colVarMap state
      varRef   = nextColVarRef state
      nvar     = asCol (ModelColVar varRef :: ModelColTerm ModelFunctions)
      colCs    = additionalCs state
      newState = state { nextColVarRef = varRef + 1
                       , colVarMap = Map.insert (getKey i) nvar varMap
                       , additionalCs = colCs ++ (newColCs nvar mcpVs) }
  put newState
  return nvar

-- Creates additional constraints for collection variables 
-- describing the size of a collection and its elements 
-- (only used for translateGecode)
newColCs :: ModelCol -> [ModelInt] -> [Model]
newColCs col vs = (size col @= cte (length vs)) : newColCs' col vs 0
  where
   newColCs' _   []     _ = []
   newColCs' col (v:vs) n = ((col @!! n) @= v) : newColCs' col vs (n+1) 

-- Translates relational operators to appropriate MCP operators
translateRelOp Equal = (@=)
translateRelOp Diff  = (@/=)
translateRelOp Less  = (@<)

-- Translates arithmetic operators to appropriate MCP operators
translateArithOp Plus  = (@+)
translateArithOp Minus = (@-)
translateArithOp Mult  = (@*)

-- ---------------------------------------------------------------------------
-- Solving MCP model
-- ---------------------------------------------------------------------------
-- Types for MCP model trees parametrized with specific FD solver:
type OvertonTree = Tree (FDInstance OvertonFD) ModelCol
type GecodeTree  = 
  Tree (FDInstance (GecodeWrappedSolver RuntimeGecodeSolver)) ModelCol

-- Calls solve function for specific solver
solveWithMCP :: MCPSolver -> MCPModel -> MCPLabelInfo -> MCPSolution
solveWithMCP Overton (ModelWrapper mcpCs) info = solveWithOverton mcpCs info
solveWithMCP Gecode  (ModelWrapper mcpCs) info = solveWithGecode  mcpCs info

-- Calls Overton Solver with corresponding model tree
-- and prepares solutions for KiCS2
solveWithOverton :: [Model] -> MCPLabelInfo -> MCPSolution
solveWithOverton mcpCs info = case maybeLabelVars of 
  Nothing -> error "MCPSolver.solveWithOverton: Found no variables for labeling."
  Just lVars -> 
    if (not (inDomain lVars dVars)) 
      then error "MCPSolver.solveWithOverton: At least for one Labeling variable
                  no domain was specified."
      else let mcpVars   = fromJust (mcpLabelVars info)
               choiceID  = fromJust (labelID info)
               strtgy    = fromJust (strategy info)
               modelTree = toModelTree mcpCs mcpVars
               solutions = snd $ solve dfs it $
                 (modelTree :: OvertonTree) >>= labelWith strtgy
           in SolWrapper (SolInfo (map (map toCurry) solutions) lVars choiceID)
  where maybeLabelVars = labelVars info
        dVars          = domainVars info

-- Calls Gecode Solver with corresponding model tree 
-- and prepares solutions for KiCS2
solveWithGecode :: [Model] -> MCPLabelInfo -> MCPSolution
solveWithGecode mcpCs info = case maybeLabelVars of 
  Nothing -> error "MCPSolver.solveWithGecode: Found no variables for labeling."
  Just lVars -> 
    if (not (inDomain lVars dVars)) 
      then error "MCPSolver.solveWithGecode: At least for one Labeling variable
                  no domain was specified."
      else let mcpVars   = fromJust (mcpLabelVars info)
               choiceID  = fromJust (labelID info)
               strtgy    = fromJust (strategy info)
               modelTree = toModelTree mcpCs mcpVars
               solutions = snd $ solve dfs it $
                 (modelTree :: GecodeTree) >>= labelWith strtgy
           in SolWrapper (SolInfo (map (map toCurry) solutions) lVars choiceID)
  where maybeLabelVars = labelVars info
        dVars          = domainVars info

-- checks, if a domain was specified for every labeling variable
inDomain :: FDList (FDTerm Int) -> [FDTerm Int] -> Bool
inDomain (FDList _ lVars) dVars = null $ lVars \\ dVars

-- Label MCP collection with given strategy
labelWith :: (FDSolver s, MonadTree m, TreeSolver m �$\sim$� FDInstance s,
  EnumTerm s (FDIntTerm s)) => LabelingStrategy -> ModelCol 
  -> m [TermBaseType s (FDIntTerm s)]
labelWith strategy col = label $ do
  lst <- getColItems col maxBound
  return $ do
    lsti <- colList col $ length lst
    labelling (matchStrategy strategy) lsti
    assignments lsti

-- select corresponding MCP labeling function for given labeling strategy
matchStrategy :: EnumTerm s t => LabelingStrategy -> [t] -> s [t]
matchStrategy FirstFail = firstFail
matchStrategy MiddleOut = middleOut
matchStrategy EndsOut   = endsOut
matchStrategy _         = inOrder

-- Transform a list of MCP constraints into a monadic MCP model tree
toModelTree :: FDSolver s => [Model] -> ModelCol -> Tree (FDInstance s) ModelCol
toModelTree model mcpLabelVars = 
  mapM_ (\m -> addC (Left m)) model >> return mcpLabelVars
\end{lstlisting}
