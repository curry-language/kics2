\section{Installation und Benutzung der KiCS2-CLPFD-Bibliothek}

\subsection{Installation}
Um die KiCS2-Finite-Domain-Constraint-Bibliothek nutzen zu k�nnen, muss man zun�chst eine KiCS2-Distribution, die diese Bibliothek beinhaltet, von der Homepage des Lehrstuhls f�r Programmiersprachen und �bersetzerkonstruktion herunterladen und installieren (Link: \href{http://www-ps.informatik.uni-kiel.de/kics2/download/}{http://www-ps.informatik.uni-kiel.de/kics2/download/}). Alternativ kann man auch eine aktuelle Version aus dem KiCS2-Repository herunterladen. Weitere Hinweise und eine Installationsanleitung findet man unter \href{http://www-ps.informatik.uni-kiel.de/kics2/repos/}{http://www-ps.informatik.uni-kiel.de/kics2/repos/} oder auch in dem entpackten KiCS2-tarfile.
\par
Nun ist es noch erforderlich, das Monadic-Constraint-Programming-Framework, das als Solver-Backend f�r die CLPFD-Bibliothek verwendet wird, zu installieren. Um auch den Gecode-Solver verwenden zu k�nnen, sollte man allerdings zun�chst die entsprechende C++-Constraint-Solver-Bibliothek installieren. Dazu l�dt man von \href{http://www.gecode.org/download.html}{http://www.gecode.org/download.html} die gepackten Quelldateien der \textbf{Gecode Version 3.1.0} herunter. Es ist wichtig, dass man genau diese Version installiert, da das Monadic-Constraint-Programming-Framework keine andere Version unterst�tzt.
\\
Zun�chst entpackt man die Quelldateien mit
\begin{lstlisting}[language=bash,frame=no]
tar xzvf gecode-3.1.0.tar.gz
\end{lstlisting}
Dann ruft man das Konfigurationsskript auf:
\begin{lstlisting}[language=bash,frame=no]
cd gecode-3.1.0/
./configure
\end{lstlisting}
Schlie�lich kann man die Quelldateien kompilieren und danach Gecode installieren:
\begin{lstlisting}[language=bash,frame=no]
make
make install
\end{lstlisting}
Nach der Installation sollte man die Gecode-Bibliotheken noch zum \emph{library path} hinzuf�gen, damit die (Beispiel-)Programme richtig gelinkt werden k�nnen.
\\
Weitere Hinweise zu Gecode kann man unter \href{http://www.gecode.org/documentation.html}{http://www.gecode.org/documentation.html} in der Anleitung \emph{Modeling and Programming with Gecode} finden.
\par
Nun kann man das Monadic-Constraint-Programming-Framework installieren. Dazu l�dt man von Hackage DB (\href{http://hackage.haskell.org/package/monadiccp-0.7.4}{http://hackage.haskell.org/package/monadiccp-0.7.4}) die neueste Version 0.7.4 (Stand: Juli 2012) des Frameworks herunter. Nachdem man die Source-Dateien entpackt hat (\lstinline|tar xzvf monadiccp-0.7.4.tar.gz|), kann man die Konfiguration durchf�hren. Dabei muss man beachten, dass das Konfigurationsflag f�r die Installation der MCP-Gecode-Quelldateien gesetzt und der Pfad zu den Bibliotheken und Header-Dateien der Gecode-Installation angegeben werden muss:
\begin{lstlisting}[language=bash,frame=no]
cd monadiccp-0.7.4
runhaskell Setup.hs configure --extra-include-dirs=
  /usr/local/include/ --extra-lib-dirs=/usr/local/lib/ 
  --flags="RuntimeGecode"
\end{lstlisting}
Nach erfolgreicher Konfiguration kann man die MCP-Pakete schlie�lich bauen und installieren:
\begin{lstlisting}[language=bash,frame=no]
runhaskell Setup.hs build
runhaskell Setup.hs install
\end{lstlisting}
Danach sollte man das Framework benutzen k�nnen. Zum Testen kann man nun eines der Beispielmodelle kompilieren und ausf�hren. Mit den folgenden Befehlen kann man beispielsweise das 4-Damen-Problem vom Gecode-Solver l�sen lassen:
\begin{lstlisting}[language=bash,frame=no]
cd examples
ghc --make Queens.hs
./Queens gecode_run 4
\end{lstlisting}
Zuerst wird also der Solver-Typ angegeben (\lstinline|gecode_run|) und dann etwaige Parameter f�r das Constraint-Problem.
\subsection{Benutzung}
Bei der Modellierung von Finite-Domain-Constraints mit der CLPFD-Bibliothek sollte man einige Dinge ber�cksichtigen:
\begin{enumerate}
\item Ein CLPFD-Modell sollte im Allgemeinen folgende Form haben:
\begin{lstlisting}[language=Haskell,frame=no]
[<unification_constraints>] & <domain_constraints> & 
  <fd_constraints> & <labeling_constraint>
\end{lstlisting}
Als erstes gibt man etwaige Curry-Gleichheits-Constraints f�r die Constraint-Variablen an, dann legt man einen Wertebereich f�r die Constraint-Variablen fest, danach folgen beliebige FD-Constraints �ber den eingef�hrten Constraint-Variablen und schlie�lich initiiert man deren Labeling durch Angabe eines "'Labeling-Constraints"'.
\item Die Curry-Gleichheits-Constraints sind optional und k�nnen gegebenenfalls auch an anderer Stelle spezifiziert werden. Sie zuerst anzugeben, kann jedoch den L�sungsvorgang beschleunigen, da diese Constraints w�hrend der Auswertung dann KiCS2-intern nicht verschoben werden m�ssen.
\item F�r das Labeling kann - falls gew�nscht - mittels des \lstinline|labelingWith|-Constraints eine spezielle Labeling-Strategie angegeben werden. Verwendet man stattdessen das \lstinline|labeling|-Constraint, so wird das Labeling der Constraint-Variablen in der gegebenen Reihefolge durchgef�hrt. Verzichtet man v�llig auf ein "'Labeling"'-Constraint, so erh�lt man eine Fehlermeldung.
\item Weiterhin ist zu beachten, dass f�r s�mtliche Labeling-Variablen auch ein Wertebereich festgelegt wird. Bei Nichtber�cksichtigung wird ebenfalls eine Fehlermeldung ausgegeben.
\end{enumerate}
