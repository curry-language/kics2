\section{Ausz�ge aus dem Monadic Constraint Programming Framework}
\label{anhangB}

\begin{lstlisting}[language=Haskell,basicstyle=\footnotesize\ttfamily,caption=Repr�sentation von Integerausdr�cken]
data Expr t c b =
    Term t
  | ExprHole Int
  | Const Integer
  | Plus (Expr t c b) (Expr t c b)
  | Minus (Expr t c b) (Expr t c b)
  | Mult (Expr t c b) (Expr t c b)
  | Div (Expr t c b) (Expr t c b)
  | Mod (Expr t c b) (Expr t c b)
  | Abs (Expr t c b)
  | At (ColExpr t c b) (Expr t c b)
  | Fold (Expr t c b -> Expr t c b -> Expr t c b) (Expr t c b) (ColExpr t c b)
  | Cond (BoolExpr t c b) (Expr t c b) (Expr t c b)
  | ColSize (ColExpr t c b)
  | Channel (BoolExpr t c b)
\end{lstlisting}

\begin{lstlisting}[language=Haskell,basicstyle=\footnotesize\ttfamily,caption=MCP \emph{collections}]
data ColExpr t c b = 
    ColTerm c
  | ColList [Expr t c b]
  | ColRange (Expr t c b) (Expr t c b)
  | ColMap (Expr t c b -> Expr t c b) (ColExpr t c b)
-- ColSlice f n c -> c[f(0)..f(n-1)]
  | ColSlice (Expr t c b -> Expr t c b) (Expr t c b) (ColExpr t c b)
  | ColCat (ColExpr t c b) (ColExpr t c b)
\end{lstlisting}

\begin{lstlisting}[language=Haskell,basicstyle=\footnotesize\ttfamily,caption=MCP-FD-Constraints]
data BoolExpr t c b =
    BoolTerm b
  | BoolConst Bool
  | BoolAnd (BoolExpr t c b) (BoolExpr t c b)
  | BoolOr (BoolExpr t c b) (BoolExpr t c b)
  | BoolNot (BoolExpr t c b)
  | BoolCond (BoolExpr t c b) (BoolExpr t c b) (BoolExpr t c b)
  | Rel (Expr t c b) ExprRel (Expr t c b)
  | BoolAll (Expr t c b -> BoolExpr t c b) (ColExpr t c b)
  | BoolAny (Expr t c b -> BoolExpr t c b) (ColExpr t c b)
  | ColEqual (ColExpr t c b) (ColExpr t c b)
  | BoolEqual (BoolExpr t c b) (BoolExpr t c b)
  | AllDiff Bool (ColExpr t c b)
  | Sorted Bool (ColExpr t c b)
  | Dom (Expr t c b) (ColExpr t c b)
\end{lstlisting}

\begin{lstlisting}[language=Haskell,basicstyle=\footnotesize\ttfamily,caption=MCP-FD-Constraint-Konstruktorfunktionen]
-- convertible to expressions:
class ToExpr tt cc bb t where
  toExpr :: t -> Expr tt cc bb

-- convertible to collection-expressions:
class ToColExpr tt cc bb c where
  toColExpr :: c -> ColExpr tt cc bb

-- convertible to boolean expressions:
class ToBoolExpr tt cc bb b where
  toBoolExpr :: b -> BoolExpr tt cc bb

(@+), (@-), (@*), (@/), (@%) :: (Eq t, Eq c, Eq b, ToExpr t c b p, ToExpr t c b q) 
                             => p -> q -> Expr t c b
a @+ b = simplify $ (toExpr a) `Plus` (toExpr b)
a @- b = simplify $ (toExpr a) `Minus` (toExpr b)
a @* b = simplify $ (toExpr a) `Mult` (toExpr b)
a @/ b = simplify $ (toExpr a) `Div` (toExpr b)
a @% b = simplify $ (toExpr a) `Mod` (toExpr b)

class (Eq tt, Eq cc, Eq bb) => ExprClass tt cc bb a where
  (@=)  :: a -> a -> BoolExpr tt cc bb
  (@/=) :: a -> a -> BoolExpr tt cc bb
  a @/= b = boolSimplify $ BoolNot $ a @= b

class (Eq tt, Eq cc, Eq bb) => ExprRange tt cc bb r where
  (@:)  :: Expr tt cc bb -> r -> BoolExpr tt cc bb

(!)   :: (Eq t, Eq c, Eq b) => ColExpr t c b -> Expr t c b -> Expr t c b
(@!!) :: (Eq t, Eq c, Eq b) => ColExpr t c b -> Integer -> Expr t c b
(@..) :: (Eq t, Eq c, Eq b) => Expr t c b -> Expr t c b -> ColExpr t c b
(@++) :: (Eq t, Eq c, Eq b) => ColExpr t c b -> ColExpr t c b -> ColExpr t c b

(@?) :: (Eq t, Eq c, Eq b) => BoolExpr t c b -> (Expr t c b, Expr t c b) 
     -> Expr t c b
c @? (t,f) = simplify $ Cond c t f

(@??) :: (Eq t, Eq c, Eq b) => BoolExpr t c b -> (BoolExpr t c b, BoolExpr t c b) 
      -> BoolExpr t c b
c @?? (t,f) = boolSimplify $ BoolCond c t f

c!p     = simplify $ At c p
c @!! p = simplify $ At c (Const p)
a @.. b = colSimplify $ ColRange (toExpr a) (toExpr b)
a @++ b = colSimplify $ ColCat (toColExpr a) (toColExpr b)

size :: (Eq t, Eq c, Eq b) => ColExpr t c b -> Expr t c b
size a = simplify $ ColSize a

xfold :: (Eq t, Eq c, Eq b) => (Expr t c b -> Expr t c b -> Expr t c b) 
      -> Expr t c b -> ColExpr t c b -> Expr t c b
xfold f i c = simplify $ Fold (\a b -> f a b) i c

xsum :: (Num (Expr t c b), Eq t, Eq c, Eq b) => ColExpr t c b -> Expr t c b
xsum c = xfold (+) (Const 0) c

list :: (Eq t, Eq c, Eq b) => [Expr t c b] -> ColExpr t c b
list x = colSimplify $ ColList x

xhead :: (Eq t, Eq c, Eq b, ToColExpr t c b p) => p -> Expr t c b
xhead c = simplify $ At (toColExpr c) (Const 0)

xtail :: (Eq t, Eq c, Eq b, ToColExpr t c b p) => p -> ColExpr t c b
xtail c = let cc = toColExpr c in colSimplify $ 
  ColSlice (\x -> simplify (x `Plus` (Const 1))) 
           (simplify $ (size cc) `Minus` (Const 1)) cc

slice :: (Eq t, Eq c, Eq b) => ColExpr t c b -> ColExpr t c b -> ColExpr t c b
slice c p = case (c,p) of
  (_,ColRange l h) -> colSimplify $ 
    ColSlice (\x -> simplify (l `Plus` x)) 
             (simplify $ Const 1 `Plus` (simplify $ h `Minus` l)) c
  (_,ColMap f (ColRange l h)) -> colSimplify $ 
    ColSlice (\i -> simplify $ f $ simplify (l `Plus` i)) 
             (simplify $ Const 1 `Plus` (simplify $ h `Minus` l)) c
  (_,ColSlice f n c2) -> colSimplify $ 
    ColSlice (\i -> simplify $ c2 `At` (f i)) n c
  _ -> xmap (\i -> simplify $ c `At` i) p

xmap :: (Eq t, Eq c, Eq b) => (Expr t c b -> Expr t c b) -> ColExpr t c b 
     -> ColExpr t c b
xmap f c = colSimplify $ ColMap f c

loopall :: (Eq t, Eq c, Eq b) => (Expr t c b,Expr t c b) 
        -> (Expr t c b -> BoolExpr t c b) -> BoolExpr t c b
loopall (l,h) f = boolSimplify $ BoolAll f $ colSimplify $ ColRange l h

loopany :: (Eq t, Eq c, Eq b) => (Expr t c b,Expr t c b) 
        -> (Expr t c b -> BoolExpr t c b) -> BoolExpr t c b
loopany (l,h) f = boolSimplify $ BoolAny f $ colSimplify $ ColRange l h

forall :: (Eq t, Eq c, Eq b) => (ColExpr t c b) 
       -> (Expr t c b -> BoolExpr t c b) -> BoolExpr t c b
forall c f = boolSimplify $ BoolAll f c

forany :: (Eq t, Eq c, Eq b) => (ColExpr t c b) 
       -> (Expr t c b -> BoolExpr t c b) -> BoolExpr t c b
forany c f = boolSimplify $ BoolAny f c

channel :: (Eq t, Eq c, Eq b) => BoolExpr t c b -> Expr t c b
channel = simplify . Channel 

inv :: (Eq t, Eq c, Eq b, ToBoolExpr t c b p) => p -> BoolExpr t c b

a @|| b = boolSimplify $ BoolOr (toBoolExpr a) (toBoolExpr b)
a @&& b = boolSimplify $ BoolAnd (toBoolExpr a) (toBoolExpr b)
inv a = boolSimplify $ BoolNot (toBoolExpr a)

(@<), (@>), (@<=), (@>=) ::  (Eq t, Eq c, Eq b) => Expr t c b -> Expr t c b 
                         -> BoolExpr t c b

a @< b = boolSimplify $ Rel a ERLess b
a @> b = boolSimplify $ Rel b ERLess a
a @<= b = boolSimplify $ Rel a ERLess (simplify $ b `Plus` (Const 1))
a @>= b = boolSimplify $ Rel b ERLess (simplify $ a `Plus` (Const 1))

sorted c = boolSimplify $ Sorted False c
sSorted c = boolSimplify $ Sorted True c
allDiff c = boolSimplify $ AllDiff False c
allDiffD c = boolSimplify $ AllDiff True c
\end{lstlisting}

\begin{lstlisting}[language=Haskell,basicstyle=\footnotesize\ttfamily,caption=\lstinline|FDSolver|-Typklasse]
class (Solver s, Term s (FDIntTerm s), Term s (FDBoolTerm s),
 Eq (FDBoolSpecType s), Ord (FDBoolSpecType s), Enum (FDBoolSpecType s), 
 Bounded (FDBoolSpecType s), Show (FDBoolSpecType s), Eq (FDIntSpecType s), 
 Ord (FDIntSpecType s), Enum (FDIntSpecType s), Bounded (FDIntSpecType s), 
 Show (FDIntSpecType s), Eq (FDColSpecType s), Ord (FDColSpecType s), 
 Enum (FDColSpecType s), Bounded (FDColSpecType s), Show (FDColSpecType s), 
 Show (FDIntSpec s), Show (FDColSpec s), Show (FDBoolSpec s)) => FDSolver s where

  -- term types
  type FDIntTerm s    :: * -- a Term of s, representing Integer variables
  type FDBoolTerm s   :: * -- a Term of s, representing Bool variables   

  -- spec types
  type FDIntSpec s    :: * -- a type specifying an Integer expression; should at
                           -- least support constant Integer's and FDIntTerm's
  type FDBoolSpec s   :: * -- a type specifying a Bool expression; should at least 
                           -- support constant Bool's and FDBoolTerm's
  type FDColSpec s    :: * -- a type specifying a Integer array expression; should
                           -- at least support lists of Int's and lists of 
                           -- IntTerm's

  -- spec type type
  type FDIntSpecType s :: *  -- a type specifying the type of an FDIntSpec s, 
                             -- in case there is more than one
  type FDBoolSpecType s :: * -- a type specifying the type of an FDBoolSpec s,
                             -- in case there is more than one
  type FDColSpecType s :: *  -- a type specifying the type of an FDColSpec s, 
                             -- in case there is more than one

  -- constructors for specifiers
  fdIntSpec_const     :: EGPar         -> (FDIntSpecType s, s (FDIntSpec s))
  fdBoolSpec_const    :: EGBoolPar     -> (FDBoolSpecType s, s (FDBoolSpec s))
  fdColSpec_const     :: EGColPar      -> (FDColSpecType s, s (FDColSpec s))
  fdColSpec_list      :: [FDIntSpec s] -> (FDColSpecType s, s (FDColSpec s))
  fdIntSpec_term      :: FDIntTerm s   -> (FDIntSpecType s, s (FDIntSpec s))
  fdBoolSpec_term     :: FDBoolTerm s  -> (FDBoolSpecType s, s (FDBoolSpec s))
  fdColSpec_size      :: EGPar         -> (FDColSpecType s, s (FDColSpec s))
  fdIntVarSpec        :: FDIntSpec s   -> s (Maybe (FDIntTerm s))
  fdBoolVarSpec       :: FDBoolSpec s  -> s (Maybe (FDBoolTerm s))

  -- function to inform about allowed types for nodes
  fdTypeReqBool :: s (EGEdge -> [(EGVarId,FDBoolSpecTypeSet s)])
  fdTypeReqInt ::  s (EGEdge -> [(EGVarId,FDIntSpecTypeSet s)])
  fdTypeReqCol ::  s (EGEdge -> [(EGVarId,FDColSpecTypeSet s)])
  fdTypeReqBool = return (\(EGEdge { egeLinks = EGTypeData { boolData = l } }) 
    -> map (\x -> (x,Set.fromList [minBound..maxBound])) l)
  fdTypeReqInt = return (\(EGEdge { egeLinks = EGTypeData { intData = l } }) 
    -> map (\x -> (x,Set.fromList [minBound..maxBound])) l)
  fdTypeReqCol = return (\(EGEdge { egeLinks = EGTypeData { colData = l } }) 
    -> map (\x -> (x,Set.fromList [minBound..maxBound])) l)

  fdTypeVarInt :: s (Set (FDIntSpecType s))
  fdTypeVarBool :: s (Set (FDBoolSpecType s))
  fdTypeVarInt = return $ Set.singleton maxBound
  fdTypeVarBool = return $ Set.singleton maxBound

  -- rating functions for specification of terms
  fdSpecify :: Mixin (SpecFn s)
  fdSpecify = mixinId

  -- inspect collections
  fdColInspect :: FDColSpec s -> s [FDIntTerm s]

  -- function to request processing an edge in a graph
  fdProcess :: Mixin (EGConstraintSpec -> FDSpecInfo s -> FDInstance s ())

  -- add equality constraints
  fdEqualBool :: FDBoolSpec s -> FDBoolSpec s -> FDInstance s ()
  fdEqualInt :: FDIntSpec s -> FDIntSpec s -> FDInstance s ()
  fdEqualCol :: FDColSpec s -> FDColSpec s -> FDInstance s ()

  fdConstrainIntTerm :: FDIntTerm s -> Integer -> s (Constraint s)
  fdSplitIntDomain :: FDIntTerm s -> s ([Constraint s],Bool)
  fdSplitBoolDomain :: FDBoolTerm s -> s ([Constraint s],Bool)
\end{lstlisting}

\begin{lstlisting}[language=Haskell,basicstyle=\footnotesize\ttfamily,caption=\lstinline|EnumTerm|-Typklasse]
class (Solver s, Term s t, Show (TermBaseType s t)) => EnumTerm s t where
  type TermBaseType s t :: *

  getDomainSize :: t -> s (Int)
  getDomain :: t -> s [TermBaseType s t]
  setValue :: t -> TermBaseType s t -> s [Constraint s]
  splitDomain :: t -> s ([[Constraint s]],Bool)
  splitDomains :: [t] -> s ([[Constraint s]],[t])
  getValue :: t -> s (Maybe (TermBaseType s t))
  defaultOrder :: [t] -> s [t]
  enumerator :: (MonadTree m, TreeSolver m ~ s) => Maybe ([t] -> m ())

  getDomainSize x = do
    r <- getDomain x
    return $ length r

  getValue x = do
    d <- getDomain x
    return $ case d of
      [v] -> Just v
      _ -> Nothing
  splitDomain x = do
    d <- getDomain x
    case d of
      [] ->  return ([],True)
      [_] -> return ([[]],True)
      _ ->   do
        rr <- mapM (setValue x) d
        return (rr,True)

  splitDomains [] = return ([[]],[])
  splitDomains (a@(x:b)) = do
    s <- getDomainSize x
    if s==0
      then return ([],[])
      else if s==1 
        then splitDomains b
        else do
          (r,v) <- splitDomain x
          if v
            then return (r,b)
            else return (r,a)

  defaultOrder = firstFail
  enumerator = Nothing
\end{lstlisting}

\begin{lstlisting}[language=Haskell,basicstyle=\footnotesize\ttfamily,caption=MCP-\lstinline|labelling|-Funktion]
labelling :: (MonadTree m, TreeSolver m ~ s, EnumTerm s t) 
          => ([t] -> s [t]) -> [t] -> m ()
labelling _ [] = true
labelling o l = label $ do 
  ll <- o l
  (cl,c) <- splitDomains ll
  let ml = map (\l -> foldr (/\) true $ map addC l) cl
  return $ do
    levelList ml
    labelling return c

levelList :: (Solver s, MonadTree m, TreeSolver m ~ s) => [m ()] -> m ()
levelList [] = false
levelList [a] = a
levelList l = let len     = length l
                  (p1,p2) = splitAt (len `div` 2) l
              in (levelList p1) \/ (levelList p2)
\end{lstlisting}

\begin{lstlisting}[language=Haskell,basicstyle=\footnotesize\ttfamily,caption=MCP-\lstinline|assignments|-Funktion]
assignment :: (EnumTerm s t, MonadTree m, TreeSolver m ~ s) 
           => t -> m (TermBaseType s t)
assignment q = label $ getValue q >>= 
  \y -> (case y of Just x -> return $ return x; _ -> return false)

assignments :: (EnumTerm s t, MonadTree m, TreeSolver m ~ s) 
            => [t] -> m [TermBaseType s t]
assignments = mapM assignment
\end{lstlisting}

\begin{lstlisting}[language=Haskell,basicstyle=\footnotesize\ttfamily,caption=MCP-Labeling-Strategien]
firstFail :: EnumTerm s t => [t] -> s [t]
firstFail qs = do ds <- mapM getDomainSize qs 
                  return [ q | (d,q) <- zip ds qs 
                             , then sortWith by d ]

inOrder :: EnumTerm s t => [t] -> s [t]
inOrder = return

middleOut :: EnumTerm s t => [t] -> s [t]
middleOut l = let n = (length l) `div` 2 in
              return $ interleave (drop n l) (reverse $ take n l)

endsOut :: EnumTerm s t => [t] -> s [t]
endsOut  l = let n = (length l) `div` 2 in
             return $ interleave (reverse $ drop n l) (take n l)

interleave []     ys = ys
interleave (x:xs) ys = x:interleave ys xs
\end{lstlisting}

\begin{lstlisting}[language=Haskell,basicstyle=\footnotesize\ttfamily,caption=MCP-Hilfsfunktionen zur Implementierung von \lstinline|labelWith|(Code-Listing 99)]
colList :: (Constraint s ~ Either Model q, MonadTree m, TreeSolver m ~ s) 
        => ModelCol -> Int -> m [ModelInt]
colList col len = do addM $ (Sugar.@=) (size col) (asExpr len)
                     return $ map (\i -> col!cte i) [0..len-1]

getColItems :: FDSolver s => ModelCol -> FDColSpecType s 
            -> FDInstance s [FDIntTerm s]
getColItems c tp = do [cc] <- getColTerm [c] tp
                      lst <- liftFD $ fdColInspect cc
                      return lst

getColTerm :: FDSolver s => [ModelCol] -> FDColSpecType s 
           -> FDInstance s [FDColSpec s]
getColTerm m tp = do
  s <- get
  put $ s { fdsForceCol = m++(fdsForceCol s) }
  commit
  s2 <- get
  let ids = map (\x -> decompColLookup (fdsDecomp s2) x) m
  specs <- mapM (\(Just id) -> getColSpec_ id (Set.singleton tp)) ids
  return $ map (snd . myFromJust ("getColTerm(tp="++(show tp)++")")) specs
\end{lstlisting}
