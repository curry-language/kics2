\section{Zusammenfassung und Ausblick}
\label{chapter_zusammenfassung}
Im Rahmen dieser Diplomarbeit wurde die KiCS2-Curry-Implementierung um die M�glichkeit zur Programmierung mit Finite-Domain-Constraints erweitert.
\\
Dazu wurde eine einfache Bibliothek zur Modellierung von Finite-Domain-Constraint-Problemen in Curry implementiert.
\\
Zur L�sung eines solchen Modells wurden die beiden FD-Solver des Monadic-Constraint-Pro\-gramming-Frameworks - der direkt in Haskell implementierte Overton-Solver sowie der auf C++-basierende Gecode-Solver - in KiCS2 integriert. Dazu wurden die FD-Constraints in semantisch �quivalente MCP-FD-Constraints �bersetzt und die resultierenden MCP-Modelle durch Aufruf eines MCP-Solvers gel�st.
\\
In einem zweiten Entwicklungsschritt wurde dann von den speziellen FD-Constraints und -Solvern abstrahiert. So wurde eine generische Schnittstelle f�r Constraints entwickelt, deren Implementierung es erm�glicht, KiCS2 um weitere Constraint-Typen zu erweitern. Des Weiteren wurde ein allgemeines Interface zur Integration von FD-Solvern in KiCS2 eingef�hrt. Diese Schnittstellen wurden f�r die FD-Constraints und die FD-Solver des MCP-Frameworks beispielhaft implementiert.
\\
Mit Hilfe von Benchmarks wurde gezeigt, dass die Verwendung der CLPFD-Bibliothek zur Modellierung von FD-Constraint-Problemen erheblich effizienter ist als ein entsprechendes Problem mit Hilfe des generate\&test-Verfahrens zu modellieren. Au�erdem erreicht man mit dieser Bibliothek trotz des zus�tzlichen �bersetzungsschritts eine ann�hernd mit der direkten L�sung durch das MCP-Framework vergleichbare Performance.
\par
F�r die Zukunft w�re es sicherlich interessant, weitere Constraint-Typen und (FD-)Solver in KiCS2 zu integrieren und auf diese Weise die hier vorgestellten Schnittstellen weiter zu testen und zu verbessern. 
\\
Au�erdem k�nnte man beim Aufruf der MCP-Solver abh�ngig von der jeweils verwendeten KiCS2-Auswertungsstrategie eine dazu passende MCP-Suchstrategie und einen geeigneten MCP-Such-Transformer einsetzen, anstatt in jedem Fall die Tiefensuche und den Identit�ts-Such-Transformer zu benutzen. 
\\
Des Weiteren k�nnte man versuchen, die FD-Constraints mit der eingekapselten Suche von KiCS2 zu kombinieren, so dass man eigene Suchstrategien �ber FD-Constraint-Modellen definieren k�nnte. Eine L�sung speziell f�r die MCP-Solver k�nnte in diesem Fall sein, die Suchstrategien mit Hilfe des MCP-Frameworks zu realisieren, das Interfaces zur Entwicklung eigener Suchstrategien und -Transformer zur Verf�gung stellt.

\cleardoublepage
